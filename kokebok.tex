\documentclass[12pt,a4paper]{book}

%% Language and font encodings
\usepackage[norsk]{babel}
\usepackage[utf8]{inputenc}
\usepackage[T1]{fontenc}

%% Sets page size and margins
%\usepackage[a4paper]{geometry}

\usepackage{graphicx}%for å kunne ha bilder i kokeboken
\usepackage{booktabs}%for fine tabeller

\usepackage{imakeidx}%for indeks bakerst

\usepackage{enumitem} %for mer kompakte lister kombinert med \begin{itemize}[noitemsep]
\usepackage[pdfauthor={Fredrik Lindseth}, pdftitle={Kokeboken til Fredrik}, pagebackref=true]{hyperref}%for \url og klikkbar innholdsfortegnelse

\usepackage{gensymb} %for \degree
\usepackage{xfrac} %for \sfrac


\hypersetup{colorlinks=true,
    colorlinks,
    citecolor=black,
    filecolor=black,
    linkcolor=black,
    urlcolor=black %fargen på link hvis colorlinks=true
    %urlbordercolor 	={0 0 1} %if colorlinks=false, ha boks i stedet
}
\urlstyle{same} %for å bruke samme font til URL som resten

\title{Kokeboken til Fredrik}
\author{Fredrik Lindseth}
\date{Sist oppdatert \today}

\makeindex

\begin{document}
\begin{titlepage}
  \maketitle

  % % front cover insted of \maketitle:
  % \thispagestyle{empty}
  % \begin{textblock*}{160mm}(-3mm,0mm)
  % %\textblockcolour{red}
  %   \includegraphics[width=160mm]{images/cover.png}  % full-page picture for 160x240 format
  % \end{textblock*}
  % \null\newpage % latex might neglect \newpage because the page is empty,


  % blank page
  % \thispagestyle{empty}
  % \null\newpage

  \setcounter{page}{5}
  \pagenumbering{roman}

\end{titlepage}


\setcounter{page}{1}
\pagenumbering{arabic}
%\include{introduction}

\chapter{Middager}
\section{Ernæringskorrekt brød}
Nok til 2 brød

\paragraph{Ingredienser}
\begin{itemize}[noitemsep]
	\item 1 l kaldt vann
	\item 12,5 gram fersk gjær (tilsvarer 1/4 pose tørrgjær)
	\item 10 gram flaksalt (for eksempel Maldon)
	\item 150 gram  linfrø
	\item 160 gram  havregryn
	\item 180 gram  siktet hvete
	\item 240 gram  sammalt hvete
	\item 240 gram  sammalt rug
\end{itemize}

\paragraph{Framgangsmåte}
\begin{enumerate}[noitemsep]
	\item Fremgangsmåte dag 1
	\begin{enumerate}[noitemsep]
		\item Bland sammen ingrediensene og elt godt (rundt 20--30 minutter i kjøkkenmaskin) til deigen henger godt sammen
		\item La deigen stå fremme i rundt en time før du setter den i kjøleskap i minimum 12 timer
	\end{enumerate}
	\item Fremgangsmåte dag 2
	\begin{enumerate}[noitemsep]

		\item Sett stekeovnen på 200 \degree~C
		\item Fordel deigen i to brødformer smurt med olje
		\item Stek brødene i cirka en time og 15--30 minutter (steketiden avhenger av stekeovnen og størrelsen på brødformene), til brødene slipper formene
		\item Legg gjerne et bakepapir over brødene halvveis i steketiden
		\item Avkjøl brødene på rist og pakk dem inn i håndklær
		\item Ikke skjær i brødene før de er ordentlig avkjølt
	\end{enumerate}
\end{enumerate}

Kilde: Oppskriften ble utviklet av to tidligere studenter på samfunnsernæring ved Høgskolen i Oslo og Akershus, Kjersti Lilleberg og Line Jensen, som en del av deres masteroppgaver.\\

Det ideelle brødet oppfyller utvalgte kriterier knyttet til både helse, smak, lukt, utseende, miljø og praktiske faktorer, og oppskriften ble dessuten utviklet på bakgrunn av litteratursøk i vitenskapelige databaser.\\

Det `ideelle' brødet har en pris på omtrent 13 kroner per brød (inkludert strømutgifter). Å kjøpe et tilsvarende brød i butikken koster imidlertid langt mer.\\

Brødet har en grovhetsgrad på 78 prosent, og fyller derfor fire felt på Brødskalaen og oppnår betegnelsen «ekstra grovt». Det oppfyller også kriteriene for Nøkkelhullet. Sist, men ikke minst inneholder brødet lite kalorier, kun 173 kcal per 100 gram brød.\\

\section{Smørdampet brokkoli}
\label{brokkoli}
En hel brokkoli til fire personer

\paragraph{Ingredienser}
\begin{itemize}[noitemsep]
	\item Brokkoli
	\item 1 ss smør
	\item Salt
\end{itemize}

\paragraph{Framgangsmåte}
\begin{enumerate}[noitemsep]
	\item Skyll brokkolien godt
	\item Kutt av den tørre enden av stilken
	\item Sett brokkolien på hodet og del den opp i spiselige biter
	\item Kok opp noen cm med vann i en panne med lokk
	\item Ha oppi litt salt og en spiseskje med smør
	\item Ligg oppi brokkolien når vannet koker
	\item Damp i 4--7 min
\end{enumerate}

Kilde: Bama: Smørdampet brokkoli for barn

\section{Brun saus}


\subsection{Brun saus ala Pappa}

\paragraph{Ingredienser}
\begin{itemize}[noitemsep]
	\item Maizenajevner (mer enn det står på pakken)
	\item Seks osthøvelslicer med brunost
	\item 1 Toro buljongterning
	\item Idun sennep
\end{itemize}


\paragraph{Framgangsmåte}
\begin{enumerate}[noitemsep]
	\item Hiv oppi alt når det koker
	\item Smak det til
\end{enumerate}

Kilde: Pappa/Frode Lindseth


\subsection{Brun saus fra Enkel Kokebok}
\paragraph{Ingredienser}
\begin{itemize}[noitemsep]
	\item 4 ss smør
	\item 5 ss hvetemel
	\item 1 terning buljong
	\item Evt 1 løk
	\item Sukkerkulør
	\item Salt og pepper
\end{itemize}


\paragraph{Framgangsmåte}
\begin{enumerate}[noitemsep]
	\item Smelt og brun smøret og tilsett hvetemel. Brun over svak varme til blandingen får en nøttebrun farge.
	\item Spe med varm kraft eller buljong og rør godt mellom hver speing.
	\item La sausen trekke under lokk ca 10 minutter
	\item Smak til med salt og pepper. Tilsett eventuellt litt løk, og om ønskelig kan sausen farges med sukkerkulør. 1 ss solbærsaft og 1 ts soyasaus gir en god saus.
	\item Server sausen ved siden av kjøttkakene eller legg kjøttkakene oppi. Kjøttkakene kan etterkokes i sausen. Da får den en fyldigere smak.
\end{enumerate}

Kilde: Enkel Kokebok

\section{Club Sandwich}
1 porsjon

\paragraph{Ingredienser}
\begin{itemize}[noitemsep]
	\item 1 kyllingfilet
	\item 3 grove brødskiver
	\item $\frac{1}{2}$ tomat
	\item agurk
	\item babyleaf-salatblanding
	\item 3 strimler med bacon
	\item grov sennep
	\item ketsjup
\end{itemize}

\paragraph{Framgangsmåte}
\begin{enumerate}[noitemsep]
	\item Kutt kyllingfileten i to
	\item Stek kyllingen, gjerne på en grill/grillpannen så den ser delikat ut
	\item Stek baconet på middels varme en god stund, så de ikke blir harde og brent
	\item Smør sennep på skivene som smør
	\item Ta en skive i bunn, ta et lag kylling, et lag bacon, en skive til, et lag kylling, bacon og en skive til slutt
	\item Salat, tomat, agurk og ketsjup etter ønske
\end{enumerate}

Tips:
Server gjerne med potetbåter

Kilde: Fritt etter Radison SAS, London.

\section{Eggerøre}
To porsjoner

\paragraph{Ingredienser}
\begin{itemize}[noitemsep]
	\item 4 stk egg
	\item 4 ss melk eller vann
	\item salt og pepper
	\item smør til steking
\end{itemize}

\paragraph{Framgangsmåte}
\begin{enumerate}[noitemsep]
	\item Visp sammen egg med melk, salt og pepper
	\item Stek røren på middels varme til det begynner å bli en fast masse.
\end{enumerate}

Tips: Eggerøren fortsetter å steke etter du har tatt den av stekepannen til den har blitt avkjølt. Stek den derfor ikke for hardt i stekepannen.

Kilde: \href{http://www.matprat.no/sunn/sunne-oppskrifter/sunne-oppskrifter-god-samvittighet/eggerore/}{}

\section{Fiskepinner}
Fire porsjoner

\paragraph{Ingredienser}
\begin{itemize}[noitemsep]
	\item 600 gram  torskefilet, uten skinn og bein
	\item 4 ss hvetemel
	\item 1 ts salt
	\item $\sfrac{1}{2}$  ts pepper
	\item 1 stk egg
	\item 2 ss melk
	\item 1 dl griljermel
	\item 3 ss margarin, flytende
\end{itemize}


\paragraph{Framgangsmåte}
\begin{enumerate}[noitemsep]
	\item Skjær torsken i ca. 2 cm brede strimler
	\item Bland hvetemel, salt og pepper i en vid skål
	\item Pisk sammen egg og melk i en annen skal
	\item Ha griljermel i en tredje vid skål
	\item Vend fiskestrimlene først i hvetemel med salt og pepper
	\item Vend dem deretter i egg- og melkeblandingen.
	\item Vend dem til slutt i griljermelet
	\item Stek de panerte fiskestrimlene i flytende margarin i varm stekepanne til de er gyllenbrune, ca. 4 minutter på hver side
	\item Legg de stekte fiskepinnene over på tørkepapir som trekker fett
\end{enumerate}

Tips: Hvis ikke man har griljermel kan man ta noen rundstykker, rive de i biter, tørke de i ovnen og knuse de for å lage sine egne smuler.

\section{Fløtegratinerte poteter}
\label{flotegratinerte}
Fire porsjoner\\
Alt er beregnet til ildfast form 17x17x5cm

\paragraph{Ingredienser}
\begin{itemize}[noitemsep]
	\item 600gram skrelte Pimpernellpoteter (ca 800gram med skall eller 9 middels poteter)
	\item	3 dl laktosefri kremfløte
	\item 1,5 dl laktosefri lettmelk
	\item 1 sølvskje kardemomme
	\item 2 sølvskjeer salt
	\item 2 sølvskjeer pepper
	\item 80 gram Norvegia
\end{itemize}

\paragraph{Framgangsmåte}
\begin{enumerate}[noitemsep]
	\item Slå på stekeovnen på 200\degree~C
	\item Skrell potetene og kutt de i tynne skiver
	\item Ta melk og fløte i en panne og hiv potetene og krydderet oppi
	\item Kok potetene i til de er ferdig kokt, ca 15 min
	\item Ta potetene og væsken over i den ildfastefile formen
	\item Rasp osten og ligg den oppå potetene
	\item Stek til osten er gyllen og fin, ca 25minutter
	\item La gjerne potene stå og kjøle seg litt før servering
\end{enumerate}

Kilde: Pappa/Frode Lindseth\\


\paragraph{Næringsinnhold per porsjon}
484 kcal\\
25\% karbohydrater, 10\% protein, 65\% fett\\
Protein 11.5g\\\
Karbohydrat 30.5g\\
Fiber 0g\\
Sukker 3.85g\\
Fett 34.8\\
Mettet fett 21.02g\\
Umettet fett 1.56\\
Kilde: Lifesum.

\include{oppskrifter/fractions}
\section{Gravet hjortekjøtt}
Passer til hvilket som helst kjøtt. F.eks mørbradbiff, men veldig veldig godt med hjort.
Oppskriften tilpasset ca 200 gram kjøtt

\paragraph{Ingredienser til marinaden}
\begin{itemize}[noitemsep]
	\item 1 ss salt
	\item 1 ss sukker
	\item 1 ts Provence
	\item 1 ts Timian
	\item 1 ts Einebær
	\item 1 ts Oregano
\end{itemize}

\paragraph{Ingredienser til tyttebærdip}
\begin{itemize}[noitemsep]
	\item Rømme
	\item Tyttebær
	\item Sukker
\end{itemize}

\paragraph{Ingredienser til sennepsdip}
\begin{itemize}[noitemsep]
	\item Rømme
	\item Dijon sennep
	\item Sukker
\end{itemize}

\paragraph{Framgangsmåte}
\begin{enumerate}[noitemsep]
	\item Bland krydderet og ta det i en brødpose
	\item Legg kjøttet inn i posen
	\item Vend posen hver gang du er i kjøleskapet
	\item La det ligge og marinere seg i tre dager mens du vender
	\item Server oppkuttet med dippene til
\end{enumerate}

\section{Guacamole}
\label{guacamole}
To porsjoner

\paragraph{Ingredienser}
\begin{itemize}[noitemsep]
	\item 2 stk avokado
	\item Chili
	\item Sitrussaft
	\item Salt og pepper
\end{itemize}


\paragraph{Framgangsmåte}
\begin{enumerate}[noitemsep]
	\item Del avocadoen i to, ta ut steinen og grav ut fruktkjøttet med en skje
	\item Mos avocadoen med en gaffel og press over sitrussaften
	\item Tilsett chili og smak til med salt og pepper og eventuelt mer limesaft
\end{enumerate}


Tips: Hvis man har mer guacamole igjen og vil ha det til neste dag. Ta alt i en boks, press det flatt og få ut luftbobler. Dekk overflaten med vann. Da er den like fin og grønn neste dag.

\section{Hamburgerbrød}
Nok til 10 hamburgerbrød

\paragraph{Ingredienser}
\begin{itemize}[noitemsep]
    \item 2,5 dl melk
    \item 1 dl vann
    \item 1 ss sukker
    \item 1 ts salt
    \item 25 gram fersk gjær
    \item 500 gram hvetemel
    \item 50 gram smør, i terninger
    \item Sesamfrø
\end{itemize}

\paragraph{Framgangsmåte}
\begin{enumerate}[noitemsep]
  \item Ha alle ingredienser til gjærdeigen, med unntak av smøret, i kjøkkenmaskinen
  \item Elt deigen i ca 10 minutter på sakte til middels fart.
  \item Tilsett smør i små terninger og elt til deigen slipper kantene i eltebollen og alt smøret er eltet godt inn i deigen, etter 5--10 minutter. Øk farten underveis.
  \item Dekk bakebollen med lokk eller plast, og sett deigen på et lunt sted og la heve til dobbel størrelse.
  \item Ha deigen på melet underlag og del den i 10 emner på ca 100 gram hver.
  \item Trill emnene til boller som du trykker flate med håndflaten
  \item La brødene etterheve i ca 30--45 minutter.
  \item Pensle brødene med lunkent vann og strø på sesamfrø
  \item Stek brødene ved ca 225 \degree~C i 10--12 minutter
\end{enumerate}

\section{Hamburgerdeig}
Fire hamburgere

% \subsection{Ala pappa}
% \paragraph{Ingredienser}
% \begin{itemize}[noitemsep]
% 	\item 500 gram Hjortehakkekjøtt
% 	\item	1 rødløk
% 	\item	Hvitløk
% 	\item	Finhakkede rødbeter
% 	\item	1 ss hvetemel
% 	\item	1 ss potetmel
% 	\item	3 egg
% 	\item	$\sfrac{1}{2}$  ts Pepper
% 	\item	2 ts Salt
% 	\item	Grillkrydder/piffi
% 	\item	Timian
% 	\item	3 einebær
% 	\item	Fløte til å spe ut med
% \end{itemize}
%
% \subsection{Vibeckeifisert utgave}
\paragraph{Ingredienser}
\begin{itemize}[noitemsep]
	\item 500 gram Hjortekjøtt
	\item	Finhakkede rødbeter
	\item	1 ss hvetemel
	\item	1 ss potetmel
	\item	3 egg
	\item	 $\sfrac{1}{2}$  ts Pepper
	\item	2 ts Salt
	\item	Grillkrydder
	\item	Timian
	\item	3 einebær
	\item	Fløte til å spe ut med
\end{itemize}

\paragraph{Framgangsmåte}
\begin{enumerate}[noitemsep]
	\item Ta kjøttdeigen i en bolle
	\item Tilsett alle krydderene og bland til alt er mikset
	\item Del alt i fire og enten lag hamburgerene med hånden eller med en burgerpresse
	\item Steik
\end{enumerate}

\section{Horn}


\paragraph{Ingredienser}
\begin{itemize}[noitemsep]
	\item 4 dl laktosefri lettmelk
	\item $\sfrac{1}{2}$ pk gjær
	\item  $\sfrac{1}{2}$  ts salt
	\item 1 ss smeltet smør
	\item 600 gram speltmel
	\item 1 egg til penslig
\end{itemize}

\paragraph{Ingredienser ganger 1.6 for å maksimere kapasiteten til kitchenaid}
\begin{itemize}[noitemsep]
	\item 6.6 dl melk
	\item 0.8gjær
	\item 0.8 ts salt
	\item 1.7ss smør
	\item 1000 gram mel
	\item 2 egg
\end{itemize}

% \paragraph{Framgangsmåte}
% \begin{enumerate}[noitemsep]
% 	\item
% \end{enumerate}

% \section{Indrefilet med sopp og gorgonzola}
% 2 porsjoner
%
% \paragraph{Ingredienser}
% \begin{itemize}[noitemsep]
% 	\item 400 gram indrefilet av storfe
% 	\item 2 Skiver bacon
% 	\item $\frac{1}{2}$  hvitløkbåt
% 	\item $\frac{1}{2}$  Kvast frisk rosmarin
% 	\item smør og olje til steking
% 	\item 100 gram østerssopp
% 	\item 20 gram gorgonzola
% 	\item $\frac{1}{4}$  dl marsala (i)
% 	\item $\frac{3}{4}$  dl frisk spinat
% 	\item $\frac{1}{2}$  ss balsamicoeddik
% 	\item $\frac{1}{2}$  ss smør
% 	\item 2 bakepoteter
% 	\item salt og pepper
%
%
% \end{itemize}
%
% \paragraph{Framgangsmåte}
% \begin{enumerate}[noitemsep]
% 	\item Skjær bakepoteter i båter. Skyll godt i kaldt vann. La renne av og evt tørk med et rent kopphåndkle
% 	\item Krydre med salt og pepper og evt paprikapulver
% 	\item Bak i ovn på 220 \degree C i 30--40 minutter
% 	\item Krydre romstemperert indrefilet, delt i porsjonsbiffer, med salt og kvern over pepper
% 	\item Rull rundt baconstrimlene og fest med en tannpirker
% 	\item Varm opp en stekepanne med litt smør og olivenolje, hvitløk og en kvast rosmarin
% 	\item Brun filetstykkene på alle sider, også langs baconstrimlene, på god varme
% 	\item Legg i ildfast fat og la hvile
% 	\item Grovhakk soppen, krydre med havsalt og kvern over pepper og la surre i godt med smør og olje – samt sjyen fra kjøttet
% 	\item Etter at filetstykkene har hvilt litt, sett i ovn på 150 \degree C i syv-åtte minutter – litt avhengig av tykkelsen på stykkene
% 	\item La hvile i ti minutters tid før servering
% 	\item Rør inn gorgonzola i soppblandingen og la surre videre sammen med marsala (eller annen hetvin) og spinat
% 	\item Legg filetstykkene på porsjonstallerkener, sammen med potetbåter Plasser soppblandingen på toppen, men la sjyen være igjen i stekepannen
% 	\item Hell marsala og balsamico i stekepannen og la koke sammen med sjyen til det er redusert med det halve
% 	\item Rør inn én spiseskje smør helt til slutt – og hell så sjyen over filetene. Server!
% \end{enumerate}
%
%
% PDFlatex sliter med URLen under.
% Kilde: \url{http://dinmat.bt.no/Finn-oppskrifter/Oppskrifter/Middag/Kj%C3%B8tt/Storfe/Indrefilet-med-sopp-og-gorgonzola/}

\section{Kjøttkaker}


\paragraph{Ingredienser}
\begin{itemize}[noitemsep]
	\item 400 gram kjøttdeig
	\item 1 ts salt
	\item 1 ts pepper
	\item $\sfrac{1}{4}$  ts muskat (valgfritt, besto bruker ikke)
	\item 2 ss potetmel
	\item 2,5 dl vann
	\item 1 egg
	\item 1 ss bakepulver
\end{itemize}

\paragraph{Framgangsmåte}
\begin{enumerate}[noitemsep]
	\item Ha kjøttdeigen i en bolle sammen med saltet
	\item Rør til deigen blir seig
	\item Tilsett krydder og potetmel og rør godt
	\item Spe med væske, litt av gangen
	\item Rør godt mellom hver speing
	\item Form til kjøttkaker som brunes i litt smør ved sterk varme og etterstekes i litt smør i ved svak varme, eller trekkes i en kjelene ved svarm varme i ca 15 minutter
	\item Store porsjoner kan etterstekes i langpannen i ovnen
	\item Server kjøttkaker med brunsaus, ertestuing, kokte poteter og tyttebærsyltetøy
\end{enumerate}


Kilde: besto/Magnhild Nordvik for ingredienser, framgangsmåte fra Enkel Kokebok

\section{Knekkebrød}
Nok til 60 knekkebrød / 3 langpanner\\
Pris 95,- (2016)

\paragraph{Ingredienser}
\begin{itemize}[noitemsep]
  \item 200 gram lettkokte havregryn
  \item 180 gram solsikkekjerner
  \item 140 gram gresskarkjerner
  \item 140 gram linfrø
  \item 110 gram sesamfrø
  \item 1 ts havsalt
  \item 7,5 dl vann
\end{itemize}

\paragraph{Framgangsmåte}
\begin{enumerate}[noitemsep]
  \item Sett stekeovnen på 160 \degree~C varmluft
  \item Mål opp og rør de tørre ingrediensene godt sammen, og tilsett deretter vannet
  \item La det stå og trekke så trekker vannet inn
  \item Rør det hele sammen, slik at alt er godt blandet til en grøtete masse
  \item Fordel røren på tre bakepapirkledde stekebrett. Bruk en slikkepott eller lignende og stryk røren jevnt utover. \emph{Dette er viktig for å få et vellykket resultat, hvis ikke kan knekkebrødene bli ujevnt stekt, og du ender opp med noen brente knekkebrød, mens resten er myke}
  \item Sett alle brettene i ovnen samtidig, og la dem først steke i ca. 10 minutter. Bruk pizzahjul eller kniv og del opp i 4×5 biter på hvert brett, tilsammen 60 knekkebrød. Sett brettene tilbake i stekeovnen og bytt plass ca hvert 15de minutt.
  \item Knekkebrødene er ferdige etter ca. 60--70 minutter total steketid, men dette varierer så her må en følge med mot slutten av steketiden.
  \item Dersom du til tross for iherdig innsats med å stryke utover massen jevn likevel har ulik tykkelse på deigen, kan det være lurt å ta ut de tynneste knekkebrødene når de er ferdige, og la de litt tykkere variantene få steke litt til, til de er ferdigstekt.
  \item Brekk knekkebrødene forsiktig fra hverandre og la dem avkjøles helt på rist.
  \item Knekkebrødene oppbevares i tett boks, og er holdbare i flere uker.
\end{enumerate}



Tips: Bruk gjerne vann med kullsyre hvis du har, da dette skal gi ekstra sprø knekkebrød. \\ Andre som ting kan brukes i oppskriften er havrekli eller fiberhusk.

\section{Koteletter i ovnen ala Nordvik}


\paragraph{Ingredienser}
\begin{itemize}[noitemsep]
	\item Koteletter
	\item Poteter
	\item Brun saus
\end{itemize}

\paragraph{Framgangsmåte}
\begin{enumerate}[noitemsep]
	\item Stek kotelettene i stekeovnen på 150 \degree~C i 2--2.5 time
	\item serveres med poteter og brun sos
\end{enumerate}

Tips: Kotelettene kan dyppes i egg og så griljermel før de puttes i ovnen for å bli enda bedre

Kilde: Magnhild

\section{Kyllingfilet med krydder og ost}


\paragraph{Ingredienser}
\begin{itemize}[noitemsep]
	\item 2 kyllingfileter uten skinn
	\item 2 Skiver  brie
	\item 1 ss frisk timian
	\item 4 Skiver bacon
	\item salt og pepper
	\item $\sfrac{1}{2}$  pk knaske røtter
	\item $\sfrac{1}{2}$  rødløk
	\item $\sfrac{1}{4}$  fennikel
	\item $\sfrac{1}{2}$  hjertesalat
	\item 2 Skiver syltede rødbeter
\end{itemize}

\paragraph{Ingredienser til urterømme}
\begin{itemize}[noitemsep]
	\item 1 dl seterrømme
	\item $\sfrac{1}{4}$  sitron
	\item $\sfrac{1}{2}$  ts salt
	\item $\sfrac{1}{4}$  ts pepper
	\item $\sfrac{1}{2}$  ss frisk dill
	\item $\sfrac{1}{2}$  ss persille
\end{itemize}

\paragraph{Framgangsmåte kyllingfilet}
\begin{enumerate}[noitemsep]
	\item Skjær et snitt i hver kyllingfilet og fyll lommen med en skive brie og litt frisk timian
	\item Ha på salt og pepper
	\item Surr et par skiver bacon rundt hver filet
	\item Stek dem litt på hver side i en middels varm stekepanne tilsatt smør eller olje, bare til filetene er pent brune
	\item Legg dem over i en ildfast form og stek videre i ovnen i 10--15 minutter, ved 175--200 \degree~C, alt etter styrken på ovnen
	\item Stek grønnsakene med litt smør eller olje til de blir knapt møre
\end{enumerate}


\paragraph{Framgangsmåte urterømme}
\begin{enumerate}[noitemsep]
	\item Press saften fra sitronen
	\item Hakk dill og persille
	\item Bland sammen rømme og urter, smak til med krydder og sitronsaft
\end{enumerate}

\section{Kyllinggrillspyd}

\paragraph{Ingredienser}
\begin{itemize}[noitemsep]
	\item Kyllingfilet i terninger
	\item Grillspyd
	\item Sesamolje
	\item Sesamfrø
	\item Aprikossyltetøy
\end{itemize}

\paragraph{Framgangsmåte}
\begin{enumerate}[noitemsep]
	\item Bland syltetøyet og peanøttsmøret sammen
	\item Tre kyllingen på grillspyd
	\item Pensle saus på kyllingen
	\item Strø sesamfrø over
	\item Stek på grillen
	\item Server med aprikossyltetøy og peanøttsmørmiksen
\end{enumerate}

\section{Kylling i tortillalefser}
\label{kyllingtortilla}

Fire porsjoner
%Hvor lang tid det tar
%Pris

\paragraph{Ingredienser}
\begin{itemize}[noitemsep]
	\item 500 gram kyllingfilet
		\item Tortillalefser, se oppskrift~\ref{tortillalefser}
		\item Norvegia
		\item 1 stk avokado
		\item Chili
		\item Honning
		\item Ruccolasalat
		\item Agurk
		\item Paprika
		\item Salt og pepper
\end{itemize}

\paragraph{Framgangsmåte}
\begin{enumerate}[noitemsep]
	\item Steik kyllingfilet og krydre med pepper og litt chili, salt
	\item Ha litt honning på kyllingen (valgfritt, brenner seg fort i pannen) og steik den litt til
	\item Tortillalefser varmes med skivet ost i ovnen til osten er smeltet (Ikke for mye da blir lefsene tørr og sprø)
	\item Ta på kylling, guacamole, salat og grønt
\end{enumerate}

Tips:
Kilde: Camilla Fjæreide

\section{Lapskaus}
%Antall porsjoner
%Hvor lang tid det tar

\paragraph{Ingredienser}
\begin{itemize}[noitemsep]
  \item 2 pakker bacon i terninger
  \item smør til steking
  \item 2 poser lapskausgrønnsaker (trenger ikke å tines)
  \item 1 liten purre
  \item 1 kvist frisk timian eller 1 ts tørket timian
  \item 1 porsjonsbeger TORO oksefond
  \item 8 dl vann
  \item 1 pose potetmospulver
  \item salt og pepper
  \item flatbrød, til servering
\end{itemize}

\paragraph{Framgangsmåte}
\begin{enumerate}[noitemsep]
  \item Smelt litt smør i ei tykkbunnet, romslig gryte, og fres baconbitene på god varme i noen minutter.
  \item Hakk purren smått og ha den i med posegrønnsakene. La alt surre i smør/baconfett på middels varme et par minutter mens du rører.
  \item Hell over vann og fond, og la det koke i fem minutter.
  \item Fisk opp timiankvistene hvis du har brukt frisk timian.
  \item Synes du lapskausen er for tynn, rører du i litt potetmospulver slik at den tykner. Spe med litt vann hvis den er for tykk.
  \item Kvern over pepper, og smak til om det trengs mer salt. Serveres gjerne med flatbrød med smør.
\end{enumerate}

\paragraph{Tips:}
\begin{itemize}[noitemsep]
  \item Tilsett terninger med persillerot og pastinakk i grønnsaksblandingen.
  \item Bruk nok pepper! Gjerne litt mer enn du tror er nødvendig, det blir som regel alltid godt.
  \item Rund av smaken med litt røkt paprika.
  \item   Bytt ut baconet med chorizo som du freser i litt varm olje og krydrer med hvitløk og chili. Det vil gi en fin og gyllen lapskaus med et krydderkick.
\end{itemize}

Kilde:\url{https://coop.no/extra/tid--penger/rask-lapskaus-med-bacon} Coop Xtra-Slik lager du god, gammeldags lapskaus på et kvarter

\section{Lasagne}


\paragraph{Ingredienser hvit saus ala Gordon Ramsey}
\begin{itemize}[noitemsep]
  \item 25 gram butter
  \item 25 gram flour
  \item 300 ml milk
  \item Pinch of ground nutmeg
  \item 60 gram cheddar cheese, grated
  \item 30 gram parmesan cheese, grated
  \item 6 sheets of `non-cook' lasagne sheets
\end{itemize}

\paragraph{Ingredienser tomatsaus ala Klikk}
\begin{itemize}[noitemsep]
  \item 1 stk tomatbokser eller tre ferske tomater
  \item 2 ss kyllingbuljong
  \item 2 stk soltørkede tomater
  \item 1 ts sukker
  \item 1 ts hvitvinseddik eller eple-eddik
  \item 1 stk laurbærblad
  \item 1 ts oregano (eller timian, merian, basilikum, estragon eller salvie)
  \item  $\sfrac{1}{2}$  ts anisfrø
  \item olje til steking
  \item salt og pepper
\end{itemize}

\paragraph{Framgangsmåte rød saus}
\begin{enumerate}[noitemsep]
  \item Ha i tomatboksen, kyllingbuljong, finhakkede soltørkede tomater, sukker, eddik og laurbærblad.
  \item Knus oregano (eller andre urter) og anisfrø i en morter (du kan droppe dette, men da blir aromaene mer dempet).
  \item Ha i kjelen. Smak til med salt og pepper.
  \item Juster også sukker/eddik etter smak.
  \item La det koke i minst 7--8 minutter.
  \item Sausen kan også koke i mye lenger på svak varme.
  \end{enumerate}

  \paragraph{Framgangsmåte hvit}
  \begin{enumerate}[noitemsep]
  \item To make the cheese sauce, first melt the butter in a saucepan.
  \item Add the flour and using a wooden spoon, stir to form a paste.
  \item Over a gentle heat add a third of the milk, whisking to prevent any lumps forming.
  \item Add the rest of the milk a third at a time, whisking as you go.
  \item Season with salt and pepper and a pinch of ground nutmeg.
  \item Allow the sauce to cook out for another minute before adding the Cheddar cheese.
  \item Stir and remove from the heat.
  \item Spoon half of the meat sauce into the bottom of the baking dish and place pasta sheets on top (break the sheets if necessary to avoid any overlapping).
  \item Next, pour in just under half of the cheese sauce, and spread evenly using a spatula before spooning the remaining meat on top.
  \item Add the final layer of pasta and use the spatula to pour over the remaining cheese sauce.
  \item Finish with the grated Parmesan and sprinkle with another pinch of oregano.
  \item Add a light seasoning of salt and pepper before cleaning the edges of the dish and placing in the oven to bake for 20--25 minutes, or until golden brown.
\end{enumerate}


Kilde: Gordon Ramsay for hvit saus. Klikk.no for tomatsausen

\section{Müsli}
Nok til ca 1kg musli.\\ Passer på en 1,75liter beholder.\\
Pris 130,- kr (2016)

\paragraph{Ingredienser}
\begin{itemize}[noitemsep]
	\item 300 gram havregryn store
	\item 170 gram solsikkekjerner
	\item 140 gram gresskarkjerner
	\item 120 gram linfrø
	\item 140 gram sesamfrø
	\item 225 gram hakkede nøtter(mandler, mandler, hasselnøtt, pecan, valnøtt, peanøtt)
	\item kardemomme
	\item kanel
	\item sesamolje
\end{itemize}

\paragraph{Framgangsmåte}
\begin{enumerate}[noitemsep]
	\item	Sett ovnen på 175 \degree C
	\item	Bland alt det tørre, hakk nøtter og bland dette sammen i en stor bolle
	\item	Tilsett kanel og kardemomme etter hvor mye smak du vil ha
	\item	Legg bakepapir på en langpanne, og fordel müsliblandingen utover
	\item	Hell olje over og vend den inn i müslien med en slikkepott
	\item	Stekes midt i ovnen på 175 \degree C i 20--25 min. Rør i blandingen underveis!
\end{enumerate}


Det er lurt å følge med under steketiden, for den kan fort bli for mye stekt.
Avkjøl den godt før du oppbevarer den i en boks.

Spises med kald melk, biola eller yoghurt naturell til. Du kan og tilsette frukt eller bær på toppen.

Kilde: Orginal fra \href{http://www.karolinegrovdal.no/?p=305}{link}, men Vibecke har modifisert

\section{Oppskriftsmal}
\label{oppskrifsmal}

Antall porsjoner på formen `fire porsjoner'\\
Hvor lang tid det tar\\
Pris

\paragraph{Ingredienser}
\begin{itemize}[noitemsep]
	\item Mengde Ingrediens
	\item 1 ss Smør (mellomrom mellom hver, stor bokstav i ingrediens og ingen punktum på slutten)
\end{itemize}

\paragraph{Framgangsmåte}
\begin{enumerate}[noitemsep]
	\item Steg1 (ingen punktum på slutten)
	\item Steg2
\end{enumerate}

Tips: Et godt tips \\

Kilde:\href{https://link}{navn på link}

\section{Panert torsk}


\paragraph{Ingredienser}
\begin{itemize}[noitemsep]
	\item Torskefilet
	\item Mel
	\item Salt og pepper
	\item Poteter
\end{itemize}

\paragraph{Framgangsmåte}
\begin{enumerate}[noitemsep]
	\item Bland mel, salt og pepper på en tallerk
	\item Dypp torsken i blandingen og hiv de på en steikepanne
	\item Steik
	\item Server med poteter
\end{enumerate}


Kilde: Mikkel

\section{Pasta al forno ala Filomena}

Fire porsjoner

\paragraph{Ingredienser}
\begin{itemize}[noitemsep]
  \item 400 gram karbonadedeig
  \item 250 gram pasta (Rigatoni)
  \item 50 gram gulrot
  \item 50 gram paprika
  \item 1 boks med tomater
  \item Så mye mozzarella som du vil. Feks 200 gram. Aller helst Scamorza.
  \item Parmesan
  \item (løk)
  \item Salt og pepper
  \item Ildfast form, 22*29*5 cm
\end{itemize}

\paragraph{Framgangsmåte}
\begin{enumerate}[noitemsep]
  \item Slå på steikeovnen på 200 \degree~C
  \item Sitt en panne med vann på kok, for pastaen
  \item Olje i steikepanne
  \item Kjøttdeig i steikepanne
  \item Rasp gulrot og hiv oppi steikepannen
  \item Kutt opp paprika og hiv oppi steikepannen
  \item Steik til kjøttet er brunt og hiv oppi boksen med tomatene
  \item Kok det opp til 1.5 time / til det er mindre flytende, men ikke for tørt. Da vil det bli for tørt i ovnen
  \item La pastaen koke, ikke så farlig om den ikke koker helt ferdig, den skal i ovnen i 30 min
  \item Start med kjøtt og tomatsaus i bunnen av den ildfaste formen og legg lagvis tomat->pasta->mozzarella
  \item Avslutt det øverste laget med parmesan
  \item Steik i ovnen 200 i 30 min eller til mozzarellaen er smeltet, vannet i bunnen er dampet litt vekk og den ser stekt ut på toppen
\end{enumerate}


Kilde: Filomena/Sergio sin mors oppskrift, via Sergio.

\section{Pasta pommedorini}
\label{pommedorini}

To porsjoner\\
20 minutter\\
Pris 50,- kr

\paragraph{Ingredienser}
\begin{itemize}[noitemsep]
	\item 150 gram farfalle. Gjerne med flere farger så det ser delikat ut
	\item 400 gram cherrytomater
	\item Olivenolje
  \item Salt og pepper
  \item Evt tørket chili
\end{itemize}

\paragraph{Framgangsmåte}
\begin{enumerate}[noitemsep]
	\item Kutt tomatene i to
	\item Ta olivenolje og salt i en steikepanne
	\item Steik tomatene til de er blitt sausete
	\item Kok pasta
	\item Hell tomatene i over pastaen og server
\end{enumerate}


Kilde: Sergio

\section{Pastasalat}
Tre porsjoner

\paragraph{Ingredienser}
\begin{itemize}[noitemsep]
	\item 450 gram kyllingfilet i terninger
	\item 200 gram pastaskruer
	\item 2 stk avokado
	\item 1 stk mango
	\item Pinjekjerner
	\item Salat (ishav, babyleaf miks, ruccola, crisp)
	\item Norvegia i terninger
\end{itemize}

\paragraph{Framgangsmåte}
\begin{enumerate}[noitemsep]
	\item Kutt kyllingen i biter
	\item Steik kyllingen
	\item Kok pasta og la den avkjøle seg litt
	\item Kutt hvitost i terninger
	\item Brun pinjekjernene i steikepannen
	\item Skjær opp mango 
	\item Kutt opp salaten
	\item Bland alt i en stor bakebolle
\end{enumerate}

\section{Pasta}
\label{pasta}
Seks porsjoner

\paragraph{Ingredienser}
\begin{itemize}[noitemsep]
  \item 5 Egg
  \item 500 gram hvetemel, gjerne type 00
\end{itemize}

\paragraph{Framgangsmåte}
\begin{enumerate}[noitemsep]
  \item Ta melet i en bolle (så slipper du at benken blir skitten)
  \item Tilsett eggene og bland det godt sammen med en gaffel
  \item Kna deigen til den er glatt og elastisk. Deigen skal ikke henge fast i hendene
  \item Plasser deigen på kjøkkenbenken kutt i fire biter
  \item Still pastamaskinen inn på instilling 0 (den med størst åpning)
  \item Kjevle pastadeigen litt flat så du får den gjennom
  \item Kjør deigen igjennom flere ganger til den har fått en rektangulær og jevn form
  \item Gradvis juster innstillingen på maskinen oppover til korrekt tykkelse for den pastaen du skal lage. Se tabell~\ref{pastatyper}
  \item Når tykkelsen er nådd, kjør pastaplaten igjennom verktøyet for å kutte den opp
  \item Kok opp en panne med rikelig med vann, for lite vann så klumper eller knekker pastaen
  \item Kok i 2--4minutter hvis pastaen er fersk eller 4--6 minutter hvis den er tørket
\end{enumerate}

Tips: Ikke bruk kalde egg rett fra kjøleskapet\\ %hvorfor ikke?
      80 gram pasta er en vanlig porsjon. 100 gram hvis man er veldig sulten.\\

For fettuccine the recommended thickness of the sheet of pasta is with the  thickness-adjustment knob on no. 5, for tagliolini it should
be on setting no. 7.
the thinnest pasta sheet thickness is achieved by setting the machine
on no. 9 and feeding the sheet of pasta through twice.\\


\begin{table}[]
\centering
\begin{tabular}{ll}
\toprule
Pastatype                            & Innstilling på Marcato Atlas \\ \midrule
Tykke nudler                         & 3                            \\
Eggnudler                            & 4                            \\
Lasagneplater, fettucine, spaghetti  & 4--5                         \\
Tortellini, tynn fettucine, linguine & 6--7                         \\
Capellini                            & 7--8                         \\ \bottomrule
\end{tabular}
\caption{Kilde: Kitchenaid Marcato Atlastilbehør}
\label{pastatyper}
\end{table}


\begin{table}[]
\centering
\begin{tabular}{ll}
\toprule
Innstilling på maskinen & Circa tykkelse på pastaplaten [mm] \\ \midrule
0                       & 4                                              \\
1                       & 3,5                                            \\
2                       & 3,2                                            \\
3                       & 2,8                                            \\
4                       & 2,5                                            \\
5                       & 2                                              \\
6                       & 1,5                                            \\
7                       & 1,3                                            \\
8                       & 1                                              \\
9                       & 0,8                                            \\ \bottomrule
\end{tabular}
\caption{Pastatykkelser}
\label{pastatykkelser}
\end{table}

Mer info om pasta \href{http://www.seriouseats.com/recipes/2015/03/uovo-in-raviolo-runny-egg-yolk-ravioli-ricotta-recipe.html}{seriouseats}

Kilde: Marcato Atlas 150 brukermanual

\section{Pepperbiff med peppersaus}
To porsjoner\\
Tid 40 min

\paragraph{Ingredienser}
\begin{itemize}[noitemsep]
	\item 350 gram  pepperbiff
	\item 1 pakke Toro Peppersaus eller lag egen\ref{peppersaus}
	\item 1 terning Kjøttbuljong
	\item 2 dl Kremfløte
	\item Salt og pepper
\end{itemize}

\paragraph{Tilbehør}
\begin{itemize}[noitemsep]
	\item Sjampinjonger (14stk små per pers)
	\item 2 dl Ris \ref{ris}
	\item Smørdampet brekkoli\ref{brokkoli}
	\item Fløtegratinerte poteter\ref{flotegratinerte}
\end{itemize}

\paragraph{Framgangsmåte}
\begin{enumerate}[noitemsep]
	\item Ta biffen ut og la den ligge i romtemperatur i 30 minutter
	\item Vask risen og la den stå og trekke mens du venter på biffen
	\item Tøm posen med saus i en panne og tøm oppi fløte, vann og buljong
	\item Kutt opp sjampinjong og ligg den på en stekepanne med smør
	\item Slå på risen og sjampinjongpannen og sausen
	\item Ta smør i stekepannen til biffen og la det bli brunt før du hiver biffene på
	\item Snu biffen ofte, krydre med salt og pepper. Stek til det er motstand i biffen og det pipler ut saft
	\item La risen koke til du ser vannet er stort sett vekke, sjekk at den er klar
\end{enumerate}

Kilde: Pappa/Frode Lindseth

Tips: Server Panna Cotta til desert\ref{pannacotta}

\section{Peppersaus}
\label{peppersaus}
Fem porsjoner

\paragraph{Ingredienser}
\begin{enumerate}[noitemsep]
  \item 8 dl stekesjy eller kraft/ buljong (pappa sin buljong + 3dl ble litt tykt)
  \item 4 ss hvetemel (Low FODMAP-mindre enn 2ss maizena?)
  \item 2 ss grønn pepper, frisk eller 1 ts syltet pepper (2ss tørket svart var mye klumper.  Prøv 1ss og kvernet)
  \item 2 dl kremfløte
  \item  $\sfrac{1}{2}$  ts salt
  \item Eventuelt 4 ss soyasaus (dropp salt hvis soya, blir salt ellers)
  \item Stekesjy fra biffen og eller pittelitt sukkerkulør for fargen
\end{enumerate}

\section{Pesto med basilikum og parmesan}
Tre porsjoner

\paragraph{Ingredienser}
\begin{itemize}[noitemsep]
	\item 75 gram basilikum
	\item 1,5 dl frisk olivenolje
	\item 50 gram pinjekjerner
	\item 100 gram parmesan
	\item 1 fedd hvitløk
	\item $\sfrac{1}{2}$ ts salt
\end{itemize}

\paragraph{Framgangsmåte}
\begin{enumerate}[noitemsep]
	\item Kjør basilikum, pinjekjerner, ost, hvitløk, salt og olje i en foodprosessor til du får en kremete masse
	      %PS: Hold av noen pinjekjerner og litt av osten.
	\item Tilsett mer olje dersom massen ikke er flytende
\end{enumerate}

Kilde \href{http://www.godt.no/o2055}{Godt.no}

\section{Pinnekjøtt}
Åtte porsjoner

\paragraph{Ingredienser}
\begin{itemize}[noitemsep]
  \item 3 kg kokefaste poteter blir ca 3 poteter per person
  \item 3 poteter ekstra til kålrabistappen
  \item 1 gulerot
  \item 4.5 kg kålrabi / ca 4 store
  \item 400 gram pinnekjøtt per pers (Handles hos meny eller Brakstads Etterfølgere (slakter på bystasjonen) eller hos slakteri i godvik)
  \item Bjørkepinner
  \item Evt desert
  \item 3dl kremfløte
  \item Tre store panner
  \item Serveringsskåler
\end{itemize}

\paragraph{Framgangsmåte --- dagen før}
\begin{enumerate}[noitemsep]
  \item Skrell potetene (ingen øyner),for at de ikke skal bli brune må de ligges i en panen og tildekket med vann
  \item Skrell kålrabien, og kutt i terninger, ligg disse også i en panne og tildekk med vann
  \item Lag klar deserten
  \item Legg noen lager med bjørkepinne i bunnen av pannen til pinnekjøttet og legg pinnekjøttet oppi. Dekk alt med vann og la det stå over natten for å unngå at det blir alt for salt
\end{enumerate}

\paragraph{Framgangsmåte --- dagen}
\begin{enumerate}[noitemsep]
  \item Om morgenen, bytt vann på pinnekjøttet. Hell av det gamle vannet og tilsett nytt.
  \item Når det nærmer seg tre timer før maten skal serveres, hell av alt vannet og hell på vann så bjørkepinnene dekkes i bunnen, men at ikke pinnekjøttet kokes.
  \item Tre timer før maten skal serveres må pannen med pinnekjøttet slåes på så de får dampe i tre timer, til kjøttet har løsnet fra beinet.
  \item Kok opp med lokket på, og skru ned så det bare akurat koker. Sørg for at man ikke tørrkoker, da blir det dårlig smak
  \item Sjekk med jevne mellomrom (VIKTIG)
  \item Øs av pinnefettet fra toppen når pinnekjøttet er ferdig og ha det i en liten panne på varme, til maten er klar for å serveres.

  \item En time før servering, Ha litt salt i vannet?  slå på platen med kålrabistappen. La den koke til kålrabien er helt mør (myk, som kokte poteter), omtrent 30 minutt.
  \item Når de er ferdigkokt: ha oppi en kokt gulrot og 2--3 poteter for en mildere smak
  \item Smak til med kremfløte og en liten ause pinnefett

  \item Potetene skal koke som vanlige poteter, i en 20 minutter, til de er mør.

  \item Server på varme tallerkner
\end{enumerate}

\subsection{tips}
Hvis man kjøper pinnekjøtt som er saget i to så får man plass til mer i pannen
Server gjerne panna cotta til desert, se oppskrift-\ref{pannacotta} for oppskrift

\subsubsection{Notater}
Julen 2016: En tredjedel av potetene var igjen, en fjerdedel av stappen og ingen kjøtt. Skulle smeltet smør til Vibecke
Tre personer brukte en time på å skrelle poteter og kålrabi for 10 personer.


\paragraph{Julebord hos Heine}
Åtte personer.
400gram pinnekjøtt * 8 personer = 3.2kg * 360kr/kg = 1150,-
500kr i kremfløte, vaniljestang og bringebær
50kr kålrabi
50kr i poteter
Totalt circa 1750,-/ 8 = 220,- per pers


Endte opp med:
500g Røykt Vossakjøt pinnekjøtt fra Meny /per person til 900.3 kr (359kr/kg)
4.5kg kålrot 44kr
3.1kg poteter 45kr
+ Pannacotta til alle
Totalt 150kr per pers


\paragraph{Et annet år}

Kålrabi 6,496 25,35,-
Vaniljestang 5 105,-
Fløte 5 stk 0,3L 89,5,-
gulrøtter 1pk 25,-
Pinnekjøtt 958,5,-
Bringebær 2 poser 29,-
Poteter 4kg 40,-
gelatin 10
Potetskreller 2 89,8,-
melis 8,-


Total
1380,15

\section{Pizzadeig}

\subsection{Napoletana fra Vinmonopolet}
til fire mindre runde pizza
\paragraph{Ingredienser}
\begin{itemize}[noitemsep]
	\item 1 kg siktet mel, helst «type 00»
	\item $\sfrac{1}{4}$  pakke fersk gjær
	\item 6 dl lunkent vann
	\item 1 ss olivenolje, virgin
	\item 1 ss salt
	\item 1 ss sukker
\end{itemize}

% \paragraph{Framgangsmåte}
% \begin{enumerate}[noitemsep]
% \end{enumerate}

Kilde \href{http://www.vinmonopolet.no/artikkel/mat-og-drikke/kombinasjoner-med-mat/pizza/drikke-til-pizza}{Vinmonopolet}


\subsection{Regal hvetemels variant}
Til en langpannepizza

\paragraph{Ingredienser}
\begin{itemize}[noitemsep]
	\item 225 gram hvetemel
	\item 1,5 dl vann
	\item $\sfrac{1}{2}$ pk gjær
	\item 1 ss rapsolje
	\item $\sfrac{1}{2}$ ts salt
	\item Annet krydder etter ønske (feks pizzakrydder)
\end{itemize}

% \paragraph{Framgangsmåte}
% \begin{enumerate}[noitemsep]
% \end{enumerate}


Kilde: Baksiden av Regal hvetemelposen

\subsection{Low FODMAP-variant}

\paragraph{Ingredienser}
\begin{itemize}[noitemsep]
	\item 2.5 dl vann
	\item 2ss olivenolje
	\item 1ts salt
	\item 1pk gjær
	\item 3dl sammalt spelt
	\item 4dl siktet spelt
\end{itemize}

% \paragraph{Framgangsmåte}
% \begin{enumerate}[noitemsep]
% \end{enumerate}


Kilde \href{http://oppskrift.klikk.no/sunn-speltpizza/2795/f}{klikk.no/sunn-speltpizza}

\section{Pizza Margarita Spechiale}


\paragraph{Ingredienser}
\begin{itemize}[noitemsep]
	\item Fersk hvit mozzarella (Aller helst cherry mozzarella/Hvite baller, ligg de oppå pizzaen)
	\item Scamorza/mozzarella underst
	\item Fersk basilikum
\end{itemize}

% \paragraph{Framgangsmåte}
% \begin{enumerate}[noitemsep]
% 	\item
% \end{enumerate}

Kilde: Pizza fra restauranten “Italia” i Sveits/Geneve/Boulevard des Philosophes

\section{Pizza Miss Italia}


\paragraph{Ingredienser}
\begin{itemize}[noitemsep]
	\item Bresaula skinke
	\item Pommedorini kuttet i to
	\item Mozzarella, fersk hvit.
	\item Svart Trøffel skivet tynne oppå
	\item Basilikum.
\end{itemize}

% \paragraph{Framgangsmåte}
% \begin{enumerate}[noitemsep]
% \item
% \end{enumerate}



Kilde: Pizza fra restauranten “Italia” i Sveits/Geneve/Boulevard des Philosophes

\section{Pappa sin potetstappe}


\paragraph{Ingredienser}
\begin{itemize}[noitemsep]
	\item 12 Poteter (melne, mandel er best, god smak)
	\item 1,5 ss Smør
	\item 2,5 dl Melk
	\item 3 omdreininger på bøssen med Salt og pepper
	\item $\sfrac{1}{8}$ ts Kardemomme
\end{itemize}

\paragraph{Framgangsmåte}
\begin{enumerate}[noitemsep]
	\item Kok potetene
	\item Skrell
	\item Kutt i biter for å gjøre mosingen lettere
	\item Mos potetene, mens du heller oppi melk til konsistensen er passelig
	\item Ta oppi 2 ss smør
	\item Smak til med salt og pepper
	\item Tuppen av en teskje med kardemomme
	\item Varmes opp igjen (ikke kok)
\end{enumerate}

\section{Råkost}


\paragraph{Ingredienser}
\begin{itemize}[noitemsep]
	\item 4 gulerøtter
	\item $\sfrac{1}{2}$  sitron
	\item 3 ss vann
	\item 1 ts sukker
\end{itemize}

\paragraph{Framgangsmåte}
\begin{enumerate}[noitemsep]
	\item Vask og skrell gulrot, og riv på et råkostjern
	\item Press saften av sitron og bland med vann og sukker
	\item Hell dressingen over gulroten og vend den godt inn
	\item Vask og finhakk gressløk, og dryss over
\end{enumerate}

\section{Ris}
\label{ris}
\index{ris}
En porsjon

\paragraph{Ingredienser}
\begin{itemize}[noitemsep]
	\item 50 gram hvis det er som tilbehør
	\item 75 gram hvis det er som hovedrett
\end{itemize}

\paragraph{Framgangsmåte}
\begin{enumerate}[noitemsep]
	\item La risen igge til bløt i pannen i 20--30min
	\item Ingen salt. Ingen smør
	\item Hvis du koker uten lokk: Kok opp vann. Tilsett ris. Kok i 10 min. Skyll i varmt vann
	\item Hvis du koker med med lokk: 175 dl vann (85gram ris * 1.75 = vann)
	\item Kok risen i ti minutter
	\item La den kjøle seg i fem minutter
\end{enumerate}

Notater: En stripete kopp i Strandgaten tar 230gram ris

\section{Speltrundstykker}
Nok til 24 rundstykker

\paragraph{Ingredienser}
\begin{itemize}[noitemsep]
	\item 550 gram siktet speltmel
	\item 250 gram sammalt grov spelt
	\item 35 gram smør
	\item  $\sfrac{1}{2}$  pk gjær
	\item 2,5 dl laktosefri melk
	\item 2,5 dl vann
	\item  $\sfrac{1}{2}$  ss salt
\end{itemize}


\paragraph{Framgangsmåte}
\begin{enumerate}[noitemsep]
	\item Lunk melk og vann til ca 37 grader og rør ut gjæren i væsken.
	\item Ha i salt og mel til en passe deig som slipper bakebollen. Underveis eltes mykt smør inn i deigen
	\item Hev til dobbel størrelse
	\item Etter heving has deigen på bakebordet. Brett den sammen et par ganger og del i passe biter.
	\item Trill ut rundstykker og sett de på ett brett.
	\item Sett alle rundstykkene inntil hverandre. Da vil de bli saftigere under stekingen
	\item Etterhev under et klede i 20 30 minutter.
	\item Kan smøres med egg eller la vær for enkelheten skyld. Hvis de smøres kan man ta sesamfrø eller sånne hvite frø.
	\item Stekes på 200 \degree~C i snaue 10 minutter. / til de har en temperatur på rett under 100 \degree~C i midten
	\item Avkjøl på rist.
\end{enumerate}

Kilde: \url{http://smakebiten.com/2012/06/29/luftige-speltrundstykker}


126 kcal/526 joule i hvert rundstykke

\section{Semulegrynsgrøt}
4 porsjoner\\
Tidsbruk: 40 minutter fra platen blir slått på til grøten er ferdig

\paragraph{Ingredienser}
\begin{itemize}[noitemsep]
	\item 1 liter melk
	\item 130 gram Semulegryn
	\item 1,5 ss smør
	\item 1 ts salt
	\item 1 ss sukker
\end{itemize}

\paragraph{Framgangsmåte}
\begin{enumerate}[noitemsep]
	\item Ta melk i pannen
	\item Tilsett semulegryn, salt og smør mens melken er kald for å unngå klumping
	\item Kok opp og la det småkoke til konsistensen til du kan skrive `ole' i grøten
	\item Server med en klatt smør, kanel og sukkers
\end{enumerate}

\section{Søtpotetfrites}
\label{frites}

2 porsjoner\\
% Hvor lang tid det tar
Pris 35,-

\paragraph{Ingredienser}
\begin{itemize}[noitemsep]
	\item 500 gram søtpotet
	\item 1 ts Maizenna maisstivelse
	\item 2 ss olivenolje
	\item salt og pepper
	\item evt timian, oregano, chili
\end{itemize}

\paragraph{Framgangsmåte}
\begin{enumerate}[noitemsep]
	\item Skrell søtpoteten, og kutt den i jevnstore fries. Det er viktig at bitene har omlag samme størrelse, slik at alle bitene blir jevnt stekt.
	\item Legg de oppkuttede søtpotetene i en bolle med kaldt vann 10--15 minutter. Dette er med på å fjerne litt av stivelsen i potetene, ifølge matbloggeren We Are Not Foodies
	\item Hell av vannet, og tørk godt av søtpotetene. Legg dem deretter i en plastikkpose med en halv teskje Maizenna, lukk posen, og rist godt så alle potetbitene er jevnt dekket av et tynt lag maisstivelse. Tanken her er at laget med maisstivelse blir sprøtt, skriver We Are Not Foodies
	\item Her er det viktig å ikke bruke for mye maisstivelse, da dette kan føre til at potetene kun smaker maisstivelse etter stekingen
	\item Åpne posen, og tilsett cirka én spiseskje olivenolje, samt ønsket krydder. Lukk posen, og rist godt så søtpotetfries-en er ordentlig dekket.
	\item Dekk et stekebrett med bakepapir, og legg søtpotetene jevnt utover. Pass på at de ikke dekker hverandre, da dette kan føre til bløte og slappe biter
	\item Sett stekebrettet inn i ovnen på 225 grader varmluft. Når bitetene begynner å se godt steikt ut må man ta brettet ut og snu bitene så alle sider blir tilstrekkelig stekt.
	\item Sett dem inn igjen i 10 nye minutter. Etter dette må man sette ovnsdøren på gløtt, og la de steke i 5--10 minutter til de har fått en sprø overflate, og et mykt og seigt indre.

\end{enumerate}

Tips:
Kilde:\href{https://coop.no/extra/mat--trender/garantert-spro-sotpotetfries/}{Cop xtra - søtpotet pommes frites}

\section{Steikt kjøttpølse}


\paragraph{Ingredienser}
\begin{itemize}[noitemsep]
	\item Kjøttpølse
	\item Potet
	\item Brun saus
\end{itemize}

\paragraph{Framgangsmåte}
\begin{enumerate}[noitemsep]
	\item Kutt kjøttpølsen i skiver
	\item skrell potet
	\item kok poteten
	\item kutt potetene i skiver
	\item steik potetskivene sammen med kjøttpølser
	\item server med brun saus
\end{enumerate}

\section{Tacokjøttboller}

\paragraph{Ingredienser}
\begin{itemize}[noitemsep]
	\item 400 gram karbonadedeig
	\item en dose tacokrydder (se oppskrift\ref{tacokrydder})
\end{itemize}

\paragraph{Framgangsmåte}
\begin{enumerate}[noitemsep]
	\item Ta tacokrydder og kjøttdeig i en bolle og bland godt
	\item Ta en spiseskje og lag boller av den krydrete kjøttdeigen
	\item Steik bollene på steikepannen til de er gjennomstekt
\end{enumerate}

\section{Tacokrydder}
\label{tacokrydder}

% \subsection{Denne vi bruker}
Nok til 60g / 7 porsjoner
\paragraph{Ingredienser}
\begin{itemize}[noitemsep]
	\item 4 ss knust tørket chili
	\item 2,5 ts tørket oregano
	\item 2 ss spisskummin
	\item 2 ss paprikapulver
	\item 1,5 ss havsalt
	\item 2 ts sort pepper (evt erstattes med 1--2 ts cayennepepper)
\end{itemize}

\paragraph{Framgangsmåte}
\begin{enumerate}[noitemsep]
	\item Bland det hele sammen og oppbevar på et tett glass. Denne mengden krydder passer fint på et 125 ml glass
	\item Doser 1 ss tacokrydder per 400 gram karbonadedeig
\end{enumerate}

Kilde: Orginaloppskrift fra Trines matblogg

% \subsection{Variant}
%
% \paragraph{Ingredienser}
% \begin{itemize}[noitemsep]
% 	\item 1 ts chilipulver
% 	\item 1 ts cayennepepper
% 	\item 1 ts chiliflak
% 	\item 1 knivsodd hvitløkspulver
% 	\item $\sfrac{1}{2}$  ts røkt paprikapulver
% 	\item $\sfrac{1}{2}$  ts vanlig søtt paprikapulver
% 	\item 1 ts svart pepper
% 	\item 1,5 ts spisskumin
% 	\item 1 ts korianderpulver
% 	\item 2 ts salt
% \end{itemize}


% Kilde: \href{http://www.godt.no/#!/artikkel/22127483/slik-lager-du-meksikansk-taco-helt-fra-bunnen-av}{Godt.no}

\section{Taco}
\label{taco}

Fire porsjoner
Hvor lang tid det tar
Pris

\paragraph{Ingredienser}
\begin{itemize}[noitemsep]
	\item 400 gram karbonadedeig
	\item 1 ss tacokrydder, se oppskrift~\ref{tacokrydder}
	\item 4 Tortillalefser, se oppskrift~\ref{tortillalefser}
	\item Tilbehør
		\begin{itemize}[noitemsep]
			\item Agurk
			\item Paprika
			\item Rømme
			\item Salsa
			\item Jalapeno
			\item Mango
			\item 1 Avokado
			\item Ruccola
			\item Norvegia
			\item Feta, soltørkede tomater og olivenolje
			\end{itemize}
\end{itemize}

\paragraph{Framgangsmåte}
\begin{enumerate}[noitemsep]
	\item Stek kjøttdeig og ta en spiseskje tacokrydder oppi kjøttet og la det trekke en stund på platen
	\item Pakk lefsene i aliminumsfolie og inn i ovnen på 200 grader i 10 minutter
	\item Lag guacamole av avokadoen, se oppskrift~\ref{guacamole}
	\item Kutt opp grønnsakene og rasp ost
\end{enumerate}

\section{Chicken tikka masala}
Fem porsjoner

\paragraph{Ingredienser til marinade}
\begin{itemize}[noitemsep]
  \item 1 dl naturell youghurt
  %\item 2 fedd hvitløk
  \item 1 ss ingefær
  \item 3 ss sitronsaft
  \item 1 ts salt
  \item $\sfrac{1}{2}$ ts chilipulver
  \item $\sfrac{3}{4}$ ts malt korianderfrø
  \item 1 ss frisk koriander
  \item 3 ts tandoori masala %15kr for 100 gram
  \item 3 ss smeltet margarin/olje
  \item 1 kg kyllingfilet
\end{itemize}

\paragraph{Ingredienser til saus}
\begin{itemize}[noitemsep]
  \item 4 ss olje
  \item  $\sfrac{1}{2}$ ts malt spisskummen
  \item $\sfrac{1}{4}$ ts javitri (muskatblomme)
  \item  $\sfrac{1}{2}$ ts jaifal (malt muskat)
  %\item 1 fedd hvitløk
  \item 1 ss ingefær
  \item 150 g knuste tomater
  \item 1 ss tomatpuré
  \item  $\sfrac{1}{2}$ ts haldi (gurkemeie)
  \item Salt og chilipulver etter smak
  \item 1 ss rømme
  \item 1 dl fløte
  \item 10 cashewnøtter
  \item 1 ss finsnittet koriander
\end{itemize}

\paragraph{Framgangsmåte marinade}
\begin{enumerate}[noitemsep]
  \item Riv ingefær på rivjern %og hvitløk
  \item Mal spisskummen og korianderfrø
  \item Del filetene i stykker på cirka 50 gram
  \item Bland alle ingrediensene og legg filetene i marinaden natten over eller minst 1 time
  \item Stekes i 10--12 minutter midt i stekeovnen på 225 \degree~C
\end{enumerate}

\paragraph{Framgangsmåte Saus}
\begin{enumerate}[noitemsep]
  \item Varm olje, ha i malt spisskummen, muskatblomme og muskat
  \item Tilsett tomater, krydder og ingefær
  \item La sausen koke i 2--3 minutter før du tilsetter rømme og fløte
  \item Bland alt godt og tilsett malte cashewnøtter og stekte kyllingfileter
  \item La de trekke i sausen i 5 min
\end{enumerate}


%Tips:
Kilde:\href{https://www.nrk.no/mat/nirus-chicken-tikka-masala-1.7314334}{Niru Kumra / NRK Matlyst}

\section{Tomatsaus}
Til pasta, pizza, etc

\paragraph{Ingredienser}
\begin{itemize}[noitemsep]
	\item 1 boks tomater
	\item 1 ts salt
	\item 1 ss sukker
	\item 1 ts pepper
	\item 1 stk laubærblad
	\item 1 ts timian
	\item $\sfrac{1}{2}$ ts anisfrø
	\item 1 ss balsamico for å nøytalisere surheten i tomater
\end{itemize}

\paragraph{Framgangsmåte}
\begin{enumerate}[noitemsep]
	\item Ta tomatene i en panne,
	\item Tilsett krydder
	\item Stavmiks det til du har en jevn blanding
	\item Smak til med krydder mens det koker
	\item La det koke til du er fornøyd med konsistensen (10 min)

\end{enumerate}

Lav FODMAP / FODMAP fri

\section{Tortillalefser}
\label{tortillalefser}
8 lefser

\paragraph{Ingredienser}
\begin{itemize}[noitemsep]
	\item 400 gram siktet speltmel
	\item 30 gram sammalt speltmel
	\item 3 dl kesam eller yoghurt naturell
	\item 2 ss olivenolje
	\item 1 ts natron
\end{itemize}

\paragraph{Framgangsmåte}
\begin{enumerate}[noitemsep]
	\item Ha halvparten av melet og natron i en bolle
	\item Ha kesam og olje oppi
	\item Kna deigen. Deigen skal være “veldig klissete”
	\item Del deigen opp i åtte biter
	\item Kjevl de flate
	\item Stek i en tørr steikepanne på høy varme til de ser fine ut
\end{enumerate}

\paragraph{Næringsinnhold}
167kcal per lefse
62% karbohydrater, 23% protein, 15% fett
Protein 9.6g
Karbohydrater 25.5g
	Fiber 1.61
	Sukker 1.67
Fett 2.7g
	Mettet fett 0.58g
	Umettet fett 2.06g
Fra Lifesum

Kilde: Thomas og Ingrid

\section{Thai Vårruller ala Dang}


\paragraph{Ingredienser}
\begin{itemize}[noitemsep]
	\item  $\sfrac{1}{2}$  Sommerkål
	\item 2 Gulerot
	\item 1 pk Glassnudler
	\item 1 Kyllingkjøttdeig
	\item 1 Egg
	\item 2 ss Østersaus. (bilde)
	\item 2 ts Buljongpulver (?)
	\item 2 ss Soya (thin boy soy sauce)
	\item 1 ss Sukker
	\item 1 pakke med vårrulldeig (Tyj spring roll pastry 50pk)
	\item Maisolje til fritering
	\item Sweetened chili sauce for spring roll
\end{itemize}

\paragraph{Framgangsmåte}
\begin{enumerate}[noitemsep]
	\item Kutt opp kålen i små biter og rasp gulrot
	\item Klipp nudlene i korte biter
	\item Stek alt i en stor stekepanne og ha  på soya, sukker og østersaus underveis
	\item Stek til alt er mykt og tørt
	\item Ligg alt i en langpanne så det får kjølt seg ned
	\item Ta eggehviten av eggeplomme og ligg til siden
	\item Ta innholdet oppå deigen og rull.  Forsegl med eggeplomme.
	\item Kok opp maisolje og friter rullene til de er gule
	\item Server med sweet chili saus
\end{enumerate}

Kilde: Dang

\section{Viltpølse med potetstappe}


\paragraph{Ingredienser}
\begin{itemize}[noitemsep]
	\item Kjøttpølse
	\item Potetstappe
	\item Tyttebærsyltetøy
	\item Dijon sennep
	      \paragraph{Til pynt}
	      \begin{itemize}[noitemsep]
	      	\item Paprika
	      	\item Ishavsalat
	      	\item Løk
	      	\item En kapers
	      	\item En oliven
	      \end{itemize}
\end{itemize}

Kilde: Servert på Muntlig på Studentsenteret



\chapter{Kaker og deserter}
\section{Bløtekake}


\paragraph{Ingredienser}
\begin{itemize}[noitemsep]
	\item 1 liter fløte
	\item Sukkerbrød, se oppskrift \ref{sukkerbrod}
	\item Noe til å pynte med. Fruktkoktail eller ferske jordbær
	\item Sitronbrus til å dynke bunnen med, evt Solo
\end{itemize}

% \paragraph{Framgangsmåte}
% \begin{enumerate}[noitemsep]
% 	\item
% \end{enumerate}

Kilde: Besten / Håkon Nordvik

\section{Kakemenn}

\subsection{Kakemenn fra Nese}

\paragraph{Ingredienser}
\begin{itemize}[noitemsep]
	\item 350 gram sukker
	\item 700 gram hvetemel (870 gram spelt)
	\item 2 dl melk
	\item 60 gram smør (margarin)
	\item 1 egg (valgfritt, men Håkon har det)
	\item 7 gram hjortesalt
	\item Et par dråper Vaniljeessens
	\item 1 ts bakepulver
\end{itemize}

\paragraph{Framgangsmåte}
\begin{enumerate}[noitemsep]
	\item Stek på 200 \degree~C til de blir litt brun i kanten
	\item Nok til fire brett med kakemenn
\end{enumerate}

Tips: Halvparten av denne mengden er nok til Vibecke og Fredrik, basert på baksten 2016.

Kilde: Oppskrift fra bakeriet på Nese via Besten/Håkon Nordvik


\subsection{Kakemenn ala Tine (med framgangsmåte)}

\paragraph{Ingredienser}
\begin{itemize}[noitemsep]
	\item 100 gram smeltet TINE Ekte Meierismør, avkjølt
	\item 3 dl sukker
	\item 2 dl TineMelk Hel
	\item 1 ts hornsalt
	\item 1 l hvetemel
\end{itemize}

\paragraph{Framgangsmåte}
\begin{enumerate}[noitemsep]
	\item Disse kakemennene blir enda bedre dersom du lar deigen skal stå kaldt en stund, gjerne natten over
	\item Rør sammen avkjølt smør, sukker, melk og halvparten av hvetemelet hvor du har blandet inn hornsalt. Tilsett mer mel til deigen er passe tykk
	\item Dekk kakemenndeigen med plast og la den stå kaldt, gjerne natten over
	\item Sett stekeovnen på 175 \degree~C og finn fram en stekeplate med bakepapir
	\item Kjevle deigen til den er ca. $\sfrac{1}{2}$ cm- tykk. Trykk ut figurer med pepperkakeformer og stek figurene i ca 7 minutter. Ta ut kakene av stekeovnen før de blir gylne
	\item La kakene avkjøles og tørke helt før de dekoreres med konditorfarge
\end{enumerate}


Kilde \href{http://www.tine.no/oppskrifter/kaker/vafler-og-smakaker/8721.cms?hvite-kakemenn-(og--damer)}{Tine.no}

\section{Karamellpudding}

\paragraph{Ingredienser}
\begin{itemize}[noitemsep]
	\item 6 dl melk
	\item 5 egg
	\item 1 ts sukker
	\item ½ ts salt
	\item ½ ts ekte vanilje (evt. kok en vaniljestang i melken)
	\item 2½ dl sukker til glasur
\end{itemize}

\paragraph{Framgangsmåte}
\begin{itemize}[noitemsep]
\item Smelt sukkeret i en stekepanne til det får en fin, brun farge. Hell det smeltede sukkeret i kakeformen. Hell det ikke bare i bunnen, men også på sidene. Det er en fordel om formen er varm. Bruk gummihansker.
\item Kok opp melken, rør i sukker, salt og vanilje. Pisk eggene såvidt sammen, bland med den avkjølte væsken. Sil blandingen opp i den glaserte formen.

\item Stivner i vannbad ved ca. 120 - 130 \textdegree{}C i ca. 1 time.

Tips: Hvis ovnen blir for varm vil puddingen heve og få en kornete konsistens.
Kilde. Håkon Lerring

\section{Lussekatter}
Nok til to brett med svære lussekatter
\subsection{Norsk}

\paragraph{Ingredienser}
\begin{itemize}[noitemsep]
  \item 50 gram TINE Ekte Meierismør
  \item 5 dl TineMelk Hel
  \item 50 gram gjær
  \item 1 gram safran
  \item 150 gram sukker
  \item $\sfrac{1}{2}$ ts salt
  \item 2 ts kardemomme
  \item 700 gram speltmel
  \item Pensling og pynt: 1 egg og 1 dl rosiner
\end{itemize}

\paragraph{Framgangsmåte}
\begin{enumerate}[noitemsep]
  \item Smelt smøret og tilsett melken. Smuldre gjæren oppi pannen med melk og smør
  \item Hvis safranen ikke er finmalt må den knuses i en mortar eller kuttes fint opp. Ellers blir ikke det skikkelig gult og du vil få røde flekker på baksten
  \item Bland sukker, mel, salt og kardemomme i en bakebokk. Bland og tilsett væslen
  \item La deigen heve under plast på et lunt sted til dobbel størrelse
  \item Sett stekeovnen på 200 \degree~C
  \item Strø litt mel på bordet og elt deigen godt
  \item Del deigen i biter og trill dem til lange fingertykke pølser og del i ca. 32 biter (16*2 brett, 4*4 lussekatter på brettet)
  \item Form lussekattene som på Tines godkjente lussekattfigurmal, figur~\ref{lussekatter}
  \item Legg de ferdig formede lussekattene på stekeplaten og la dem heve under plast på et lunt sted i ca. 15 minutter
  \item Pensle med sammenvispet egg og pynt med rosiner
  \item Stek lussekattene på 200 grader i  ca 10 minutter
  \item La de avkjøles litt på rist før servering. De smaker aller best litt lunkne
\end{enumerate}

Tips: Halvparten av denne mengden er nok til Vibecke og Fredrik, basert på baksten 2016.

Kilde: Orginalt fra \href{http://www.tine.no/oppskrifter/bakst/sot-gjarbakst/lussekatter}{Tine.no}, kraftig modifisert


\subsection{Français}
\paragraph{Ingrédients}
\begin{itemize}[noitemsep]
  \item 50g de beurre
  \item 5 décilitre de lait entier
  \item 50g de levure
  \item 1g de safran
  \item 150g de sucre
  \item 1/2 cuillére à café de sel
  \item 2 cuillére à café de cardamome
  \item 13 décilitre de farine
  \item 1 eouf
  \item 1 décilitre de raisins secs
\end{itemize}

\paragraph{Préparition}
\begin{enumerate}[noitemsep]
  \item Fait fondre le beurre. Ajouter le lait. Émiette la levure dans un bol et ajouter le lait et le beurre
  \item Écraser le safran dans un mortier et l'ajouter  dans le bol
  \item Ajouter le sucre, le sel, la cardamome, le safran et la farine jusqu'à ce que pâte soit ferme
  \item Laisser lever la pâte jusqu'à c'est doublé de taille.
  \item Pétrir la pâte et faire des saucisses de l'epaisseur d'un doigt. Divisé en 30 morceaux
  \item Façonner la pâte dans les formes vous voulez, laissez-les reposes pendant 15 minutes et bedigeonner avec l'oeuf et décorer avec des raisins secs
  \item Faire cuire dans le four á 250 degrés pour 5--8 minutes.
  \item Refroidir sur une grille
  \item Voila
\end{enumerate}


\href{https://fr.wikipedia.org/wiki/Lussekatt}{fr.wikipedia/lussekatt}


\begin{figure}[p]
\label{lussekatter}
  \includegraphics[width=\textwidth]{bilder/lussekatter.png}
  \caption[Lussekatter]{Lussekatter. Kilde \href{http://www.tine.no/imageresize/383493_999_1150.png}{Tine.no}}
\end{figure}

\section{Ostekake}


\paragraph{Ingredienser}
\begin{itemize}[noitemsep]
	\item 250 gram Digestive kjeks
	\item 110 gram smeltet margarin
	\item 1 pakke Freia Gele
	\item 1 pakke Philadelphia mykost
	\item 110 gram melis røres inn
	\item 1 ts vaniljesukker
	\item 1 boks med lettrømme
	\item 3 dl kremfløte pisket til krem
\end{itemize}

\paragraph{Framgangsmåte}
\begin{enumerate}[noitemsep]
	\item Bland sammen kjeksene og smeltet margarin
	\item ligg blandingen i bunnen av en sprinform
	\item sitt det i kjøleskapet
	\item 1 pk med Freia Gele, men bruk halvparten av vannet som pakken sier
	\item Avkjøles Pitte litt
	\item 1 pk Philadelphia naturell røres inn
	\item 110 gram melis røres inn
	\item 1 ts vaniljesukker
	\item 1 boks med lettrømme
	\item 3 dl kremfløte pisket til krem
	\item Bland alt
	\item Ligg i formen oppå bunnen
	\item Dagen etter elns.  Midten må stivne
	\item Gelé med halvparten av vannet
	\item Avkjøl litt og hell oppå midten
\end{enumerate}

Kilde: Trine Nordvik

\section{Panna cotta}
\label{pannacotta}
To porsjoner

\paragraph{Ingredienser}
\begin{itemize}[noitemsep]
	\item 4 gram eller 2 stk gelatinplater
	\item 1 stk vaniljestang
	\item 3 dl kremfløte %18 kr 2016
	\item 90 gram sukker
	\item 50 gram frosne bringebær
	\item 25 gram melis
	\item 1 ss sitronsaft
\end{itemize}

\paragraph{Framgangsmåte}
\begin{enumerate}[noitemsep]
	\item Legg gelatinplatene i kaldt vann i minst 5 minutter
	\item Snitt vaniljestangene på midten, skrap ut frøene og opp i pannen
	\item Kok forsiktig opp fløte, sukker, hele vaniljestangen og innholdet i stangen
	\item Skyll gelatinen godt og klem ut vannet
	\item Ta det av platen, ta ut vaniljestangen, og rør inn gelatinen
	\item Hell blandingen i to små former eller glass
	\item Avkjøl minst 3 timer før servering
	\item Ha tinte bringebær, melis og sitronsaft i en foodprocessor og kjør til alt er most.
	\item Smak til med sitron og melis
\end{enumerate}


Tips: Kan fint lages en til to dager før servering. Server deserten gjerne i vinglass. Bruk trakt og blokker åpningen med fingen for å unngå gris på glassene når du heller oppå.\\

Kilde: \href{http://www.matprat.no/gjester/gjesteoppskrifter/panna-cotta/}{Matprat.no/Panna cotta}

\section{Pepperkaker}


\paragraph{Ingredienser}
\begin{itemize}[noitemsep]
	\item 150 gram TINE Ekte Meierismør
	\item 1 dl sirup
	\item 2 dl sukker
	\item 1 dl TINE Kremfløte
	\item $\sfrac{1}{2}$ ts nellik
	\item $\sfrac{1}{2}$ ts ingefær
	\item $\sfrac{1}{2}$ ts pepper
	\item 2 ts kanel
	\item 1 ts bakepulver
	\item 450 gram hvetemel
\end{itemize}

\paragraph{Framgangsmåte}
\begin{enumerate}[noitemsep]
	\item Sett stekeovnen på 175 \degree~C
	%Til denne oppskriften trenger du pepperkakeformer
	\item Bland smør, sirup, sukker i en kjele
	\item Varm opp på middels varme til alt er helt smeltet
	\item Ta kjelen av platen og avkjøl blandingen noe
	\item Rør i fløten. Sikt i krydder, bakepulver og det meste av melet
	\item Rør alt sammen til en jevn deig og la deigen stå kaldt til neste dag
	\item Elt deigen i litt mel på bordet og kjevle den ca. 3 mm tykk
	\item Stikk ut forskjellige figurkaker og stek dem i 9--10 minutter til de er gyllenbrune.
	\item Avkjøl kakene på rist
	\item Pynt eventuelt pepperkakene med melisglasur
\end{enumerate}



Tips: Husk at pepperkakedeigen må ligge kaldt en stund før du kjevler og stikker den ut, så det beste er om du lager den dagen før.\\

Kilde \url{http://www.tine.no/oppskrifter/kaker/6987.cms?julens-deiligste-pepperkaker}

\section{Riskrem og rød saus}
\label{riskrem}

%Antall porsjoner på formen `fire porsjoner'
Tid: 10 minutter
%Pris

\paragraph{Ingredienser riskrem}
\begin{itemize}[noitemsep]
	\item Risengrynsgrøt
	\item Kremfløte
	\item Sukker
\end{itemize}

\paragraph{Ingredienser rød saus}
\begin{itemize}[noitemsep]
	\item 1 dl sukker
	\item 1 dl vann
	\item en pose frosne bringebær eller jordbær
\end{itemize}

\paragraph{Framgangsmåte}
\begin{enumerate}[noitemsep]
	\item Visp fløten
	\item Bland inn kremen i grøten
\end{enumerate}

\paragraph{Framgangsmåte rød saus}
\begin{enumerate}[noitemsep]
	\item Kok opp vann og ha i sukker
	\item Rør til sukkeret har løst seg opp i vannet
	\item Tilsett bær og stavmiks det hele til det er en jevn blanding
	\item Evt ha det igjennom en sil for å gjøre blandingen enda jevnere
\end{enumerate}




Tips:
Kilde: Solveig

\section{Sjokoladekjeks}
Nok til 40--50 `vanlig store' kjeks

\paragraph{Ingredienser}
\begin{itemize}[noitemsep]
	\item 6 dl hvetemel
	\item 2,5 dl brunt sukker
	\item 1 dl farin
	\item 200 gram margain
	\item 200 gram melkesjokolade
	\item 2 ts vaniljesukker
	\item 1 ts natron
	\item 15 dråper vaniljeessens
\end{itemize}

\paragraph{Framgangsmåte}
\begin{enumerate}[noitemsep]
	\item Bland alt
	\item Stek på 190 \degree C i 7--9 minutter
\end{enumerate}

Tips: Ta ut margarinet på benken lenge før så det blir mykt

Kilde: Ingrid 12HBIO

\section{Håkons sukkerbrød}
\label{sukkerbrod}

\paragraph{Ingredienser}
\begin{itemize}[noitemsep]
	\item 100 gram sukker
	\item 100 gram hvetemel
	\item 1 ts bakepulver
	\item 4 egg
\end{itemize}

\paragraph{Framgangsmåte}
\begin{enumerate}[noitemsep]
	\item Slå på ovnen på 175 \degree~C
	\item Smør springformen med olivenolje
	\item Tilsett egg og sukker i Kenwooden
	\item Pisk sammen til det blir hvitt og stivt
	\item Bland bakepulveret i melet
	\item Tilsett hvetemel og bakepulver mens maskinen går
	\item Ta bollen av maskinen og bland for hånd til alt hvetemelet fra kanten er blandet inn
	\item Stek i 30 min
\end{enumerate}


Tips
\begin{itemize}[noitemsep]
	\item Snu formen på hodet når du tar den ut av ovnen så den ikke skal synke sammen.
	\item Ta i mer mel hvis det er små egg, så den ikke synker sammen like lett.\\
\end{itemize}

Kilde: Etter oppskrift fra Severin og bakeriet på Nese via Besten/Håkon Nordvik

\section{Tyskeskjeva}


\paragraph{Ingredienser}
\begin{itemize}[noitemsep]
	\item 500 gram melis
	\item 750 gram margarin
	\item 1 kg hvetemel
\end{itemize}

\paragraph{Framgangsmåte}
\begin{enumerate}[noitemsep]
	\item Bland alt sammen
	\item Kna deig ut til en lang pølse
	\item La den stå over natten
	\item Skjær i tynne skiver ca 0.5 cm
	\item Steik til litt brun i kanten 200\degree~C
\end{enumerate}

Bakeriet på Nese via Håkon Nordvik



\chapter{Brygg}
\input{sorterteBrygg}

\printindex


\end{document}
