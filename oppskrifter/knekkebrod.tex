\section{Knekkebrød}
Nok til 60 knekkebrød / 3 langpanner\\
Pris 95,- (2016)

\paragraph{Ingredienser}
\begin{itemize}[noitemsep]
	\item 200 gram lettkokte havregryn
	\item 180 gram solsikkekjerner
	\item 140 gram gresskarkjerner
	\item 140 gram linfrø
	\item 110 gram sesamfrø
	\item 1 ts havsalt
	\item 7,5 dl vann
\end{itemize}

\paragraph{Framgangsmåte}
\begin{enumerate}[noitemsep]
	\item Sett stekeovnen på 160 \degree C varmluft
	\item Mål opp og rør de tørre ingrediensene godt sammen, og tilsett deretter vannet
	\item La det stå og trekke så trekker vannet inn
	\item Rør det hele sammen, slik at alt er godt blandet til en grøtete masse
	\item Fordel røren på tre bakepapirkledde stekebrett. Bruk en slikkepott eller lignende og stryk røren jevnt utover. \emph{Dette er viktig for å få et vellykket resultat, hvis ikke kan knekkebrødene bli ujevnt stekt, og du ender opp med noen brente knekkebrød, mens resten er myke}
	\item Sett alle brettene i ovnen samtidig, og la dem først steke i ca. 10 minutter. Bruk pizzahjul eller kniv og del opp i 4×5 biter på hvert brett, tilsammen 60 knekkebrød. Sett brettene tilbake i stekeovnen og bytt plass ca. hvert 15/20 minutt.
	\item Knekkebrødene er ferdige etter ca. 60--70 minutter total steketid, men dette varierer så her må en følge med mot slutten av steketiden.
	\item Dersom du til tross for iherdig innsats med å stryke utover massen jevn likevel har ulik tykkelse på deigen, kan det være lurt å ta ut de tynneste knekkebrødene når de er ferdige, og la de litt tykkere variantene få steke litt til, til de er ferdigstekt.
	\item Brekk knekkebrødene forsiktig fra hverandre og la dem avkjøles helt på rist.
	\item Knekkebrødene oppbevares i tett boks, og er holdbare i flere uker.
\end{enumerate}



Tips: Bruk gjerne vann med kullsyre hvis du har, da dette skal gi ekstra sprø knekkebrød. \\ Andre som ting kan brukes i oppskriften er havrekli eller fiberhusk.
