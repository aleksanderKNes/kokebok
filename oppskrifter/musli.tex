\section{Müsli}
Nok til ca 1,75liter

\paragraph{Ingredienser}
\begin{itemize}[noitemsep]
	\item 5 dl havregryn store
	\item 1 dl solsikkekjerner
	\item 1 dl gresskarkjerner
	\item 1 dl linfrø
	\item 1 dl sesamfrø
	\item 3dl hakkede nøtter (peanutter, pistasj, pecan, mandler, valnøtt)
	\item hakket mandler og valnøtter
	\item rosiner (etter eget ønske- blandes inn når den er ferdig)
	\item litt kardemomme og kanel
	\item 3 ss kokosolje
	\item 3 ss honning
\end{itemize}

\paragraph{Framgangsmåte}
\begin{enumerate}[noitemsep]
	\item	Sett ovnen på 175 \degree C
	\item	Bland alt det tørre, grov hakk mandler og valnøtter og bland dette sammen i en stor bolle
	\item	Tilsett kanel og kardemomme etter hvor mye smak du vil ha
	\item	Smelt kokosolje og honning sammen i en kjele på svak varme, til det er flytende
	\item	Legg bakepapir på en langpanne, og fordel müsli blandingen utover
	\item	Hell kokosolje og honningen over og vend den inn i müslien med en slikkepott
	\item	Stekes midt i ovnen på 175 \degree C i ca.15--20min. Rør i blandingen underveis!
\end{enumerate}


Det er lurt å følge med under steketiden, for den kan fort bli for mye stekt.
Avkjøl den godt før du oppbevarer den i en boks.

Spises med kald melk, biola eller yoghurt naturell til. Du kan og tilsette frukt eller bær på toppen.

%Kilde: \href{http://www.karolinegrovdal.no/?p=305}{link}
