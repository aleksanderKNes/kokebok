\section{Müsli}
Nok til ca 1kg musli.\\ Passer på en 1,75liter beholder.\\
Pris 130,- kr (2016)

\paragraph{Ingredienser}
\begin{itemize}[noitemsep]
  \item 300 gram havregryn store
  \item 170 gram solsikkekjerner
  \item 140 gram gresskarkjerner
  \item 120 gram linfrø
  \item 140 gram sesamfrø
  \item 225 gram hakkede nøtter (mandler, mandler, hasselnøtt, pecan, valnøtt, peanøtt)
  \item kardemomme
  \item kanel
  \item sesamolje
\end{itemize}

\paragraph{Framgangsmåte}
\begin{enumerate}[noitemsep]
  \item	Sett ovnen på 175 \degree~C
  \item	Bland alt det tørre, hakk nøtter og bland dette sammen i en stor bolle
  \item	Tilsett kanel og kardemomme etter hvor mye smak du vil ha
  \item	Legg bakepapir på en langpanne, og fordel müsliblandingen utover
  \item	Hell olje over og vend den inn i müslien med en slikkepott
  \item	Stekes midt i ovnen på 175 \degree~C i 20--25 min. Rør i blandingen underveis!
\end{enumerate}


Det er lurt å følge med under steketiden, for den kan fort bli for mye stekt.
Avkjøl den godt før du oppbevarer den i en boks.

Spises med kald melk, biola eller yoghurt naturell til. Du kan og tilsette frukt eller bær på toppen.

Kilde: Orginal fra \href{http://www.karolinegrovdal.no/?p=305}{www.karolinegrovdal.no}
