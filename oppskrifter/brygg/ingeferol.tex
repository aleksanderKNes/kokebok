\section{Ingefærøl}


\paragraph{Ingredienser til 19L}
\begin{itemize}[noitemsep]
	\item 907 gram ingefær, kuttet i skiver
	\item 3.4kg sukker
	\item 2 kanelstenger
	\item 3 nellikstenger
	\item 1 vaniljestang
	\item 3.5 teskje allehånde
	\item 1 pakke champagne gjær
\end{itemize}

\paragraph{Ingredienser, skalert for å passe på en 5L glassballong}
\begin{itemize}[noitemsep]
	\item 225gram ingefær
	\item 850 gram sukker
	\item ½ kanelstang
	\item ¾ nellik
	\item 0.8 teskje allehånde
	\item ¼ vaniljestang
	\item champagnegjær
\end{itemize}

\paragraph{Framgangsmåte}
\begin{enumerate}[noitemsep]
	\item Kok ingefæreren i 30 minutt og fjern fra varmen
	\item Bland inn suger og krydder og la det kjøle seg
	\item Overfør det til fermentereren og tilsett gjær

	\item After fermentation is complete, rack to a secondary, crush and add 5 campden tablets to kill any remaining yeast.
	\item You can also add another lb of fresh ginger if you really like the ginger flavor.
	\item Let it sit for about 2 weeks (1 week if not using more ginger).
	\item Boil a small amount of water and add sugar needed for desired sweetness.
	\item Add to keg, then rack the ginger beer on top. Force carbonate at 30 PSI (serving pressure).
\end{enumerate}




Kilde: \url{http://www.homebrewtalk.com/showthread.php?t=91970}

\paragraph{Notater}

OG 1062
Dry english ale WLP007
Brygget 2016--03--29

2016--04--03
Brukte hevert, tok det over i en annen glassballong. Silte av ingefær og store krydder og bunnfall.
Gravity: 1052

2016--04--05
Smakte to glass. Veldig søtt, ikke så mye kullsyre. Hint av gjær?

2016--04--09
Stakk det over på en ny ballong. En god del bunnfall av gjær.
Gravity: 1049  1.71prosent alkohol.
Fortsatt litt uklar. Hint av gjær i smaken.
Satt ballongen inn på soverommet på gulvet i kulden

2016--04--10
Væsken var veldig klar. Mer bunnfall.
Smaker godt, søtt, sterkt. Ikke noe gjærsmak. Lysere i fargen, mer ingefær smak enn crabbies.
