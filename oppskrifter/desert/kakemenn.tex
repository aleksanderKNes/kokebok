\section{Kakemenn}

\subsection{Kakemenn fra Nese}

\paragraph{Ingredienser}
\begin{itemize}[noitemsep]
	\item 350 gram sukker
	\item 700 gram hvetemel (870 gram spelt)
	\item 2 dl melk
	\item 60 gram smør (margarin)
	\item 1 egg (valgfritt, men Håkon har det)
	\item 7 gram hjortesalt
	\item Et par dråper Vaniljeessens
	\item 1 ts bakepulver
\end{itemize}

\paragraph{Framgangsmåte}
\begin{enumerate}[noitemsep]
	\item Stek på 200 \degree~C til de blir litt brun i kanten
	\item Nok til fire brett med kakemenn
\end{enumerate}

Tips: Halvparten av denne mengden er nok til Vibecke og Fredrik, basert på baksten 2016.

Kilde: Oppskrift fra bakeriet på Nese via Besten/Håkon Nordvik


\subsection{Kakemenn ala Tine (med framgangsmåte)}

\paragraph{Ingredienser}
\begin{itemize}[noitemsep]
	\item 100 gram smeltet TINE Ekte Meierismør, avkjølt
	\item 3 dl sukker
	\item 2 dl TineMelk Hel
	\item 1 ts hornsalt
	\item 1 l hvetemel
\end{itemize}

\paragraph{Framgangsmåte}
\begin{enumerate}[noitemsep]
	\item Disse kakemennene blir enda bedre dersom du lar deigen skal stå kaldt en stund, gjerne natten over
	\item Rør sammen avkjølt smør, sukker, melk og halvparten av hvetemelet hvor du har blandet inn hornsalt. Tilsett mer mel til deigen er passe tykk
	\item Dekk kakemenndeigen med plast og la den stå kaldt, gjerne natten over
	\item Sett stekeovnen på 175 \degree~C og finn fram en stekeplate med bakepapir
	\item Kjevle deigen til den er ca. $\sfrac{1}{2}$ cm- tykk. Trykk ut figurer med pepperkakeformer og stek figurene i ca 7 minutter. Ta ut kakene av stekeovnen før de blir gylne
	\item La kakene avkjøles og tørke helt før de dekoreres med konditorfarge
\end{enumerate}


Kilde \href{http://www.tine.no/oppskrifter/kaker/vafler-og-smakaker/8721.cms?hvite-kakemenn-(og--damer)}{Tine.no}
