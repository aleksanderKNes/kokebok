\section{Pepperkaker}


\paragraph{Ingredienser}
\begin{itemize}[noitemsep]
	\item 150 gram TINE Ekte Meierismør
	\item 1 dl sirup
	\item 2 dl sukker
	\item 1 dl TINE Kremfløte
	\item $\sfrac{1}{2}$ ts nellik
	\item $\sfrac{1}{2}$ ts ingefær
	\item $\sfrac{1}{2}$ ts pepper
	\item 2 ts kanel
	\item 1 ts bakepulver
	\item 450 gram hvetemel
\end{itemize}

\paragraph{Framgangsmåte}
\begin{enumerate}[noitemsep]
	\item Sett stekeovnen på 175 \degree~C
	%Til denne oppskriften trenger du pepperkakeformer
	\item Bland smør, sirup, sukker i en kjele
	\item Varm opp på middels varme til alt er helt smeltet
	\item Ta kjelen av platen og avkjøl blandingen noe
	\item Rør i fløten. Sikt i krydder, bakepulver og det meste av melet
	\item Rør alt sammen til en jevn deig og la deigen stå kaldt til neste dag
	\item Elt deigen i litt mel på bordet og kjevle den ca. 3 mm tykk
	\item Stikk ut forskjellige figurkaker og stek dem i 9--10 minutter til de er gyllenbrune.
	\item Avkjøl kakene på rist
	\item Pynt eventuelt pepperkakene med melisglasur
\end{enumerate}



Tips: Husk at pepperkakedeigen må ligge kaldt en stund før du kjevler og stikker den ut, så det beste er om du lager den dagen før.\\

Kilde \url{http://www.tine.no/oppskrifter/kaker/6987.cms?julens-deiligste-pepperkaker}
