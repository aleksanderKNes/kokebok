\section{Pasta}
Seks porsjoner

\paragraph{Ingredienser}
\begin{itemize}[noitemsep]
  \item 5 Egg
  \item 500 gram hvetemel, gjerne type 00
\end{itemize}

\paragraph{Framgangsmåte}
\begin{enumerate}[noitemsep]
  \item Ta melet i en bolle (så slipper du at benken blir skitten)
  \item Tilsett eggene og bland det godt sammen med en gaffel
  \item Kna deigen til den er glatt og elastisk. Deigen skal ikke henge fast i hendene
  \item Plasser deigen på kjøkkenbenken kutt i fire biter
  \item Still pastamaskinen inn på instilling 0 (den med størst åpning)
  \item Kjevle pastadeigen litt flat så du får den gjennom
  \item Kjør deigen igjennom flere ganger til den har fått en rektangulær og jevn form
  \item Gradvis juster innstillingen på maskinen oppover til korrekt tykkelse for den pastaen du skal lage. Se tabell~\ref{pastatyper}
  \item Når tykkelsen er nådd, kjør pastaplaten igjennom verktøyet for å kutte den opp
  \item Kok opp en panne med rikelig med vann, for lite vann så klumper eller knekker pastaen
  \item Kok i 2--4minutter hvis pastaen er fersk eller 4--6 minutter hvis den er tørket
\end{enumerate}

Tips: Ikke bruk kalde egg rett fra kjøleskapet\\ %hvorfor ikke?
      80 gram pasta er en vanlig porsjon. 100 gram hvis man er veldig sulten.\\

For fettuccine the recommended thickness of the sheet of pasta is with the  thickness-adjustment knob on no. 5, for tagliolini it should
be on setting no. 7.
the thinnest pasta sheet thickness is achieved by setting the machine
on no. 9 and feeding the sheet of pasta through twice.\\


\begin{table}[]
\centering
\begin{tabular}{ll}
\toprule
Pastatype                            & Innstilling på Marcato Atlas \\ \midrule
Tykke nudler                         & 3                            \\
Eggnudler                            & 4                            \\
Lasagneplater, fettucine, spaghetti  & 4--5                         \\
Tortellini, tynn fettucine, linguine & 6--7                         \\
Capellini                            & 7--8                         \\ \bottomrule
\end{tabular}
\caption{Kilde: Kitchenaid Marcato Atlastilbehør}
\label{pastatyper}
\end{table}


\begin{table}[]
\centering
\begin{tabular}{ll}
\toprule
Innstilling på maskinen & Circa tykkelse på pastaplaten [mm] \\ \midrule
0                       & 4                                              \\
1                       & 3,5                                            \\
2                       & 3,2                                            \\
3                       & 2,8                                            \\
4                       & 2,5                                            \\
5                       & 2                                              \\
6                       & 1,5                                            \\
7                       & 1,3                                            \\
8                       & 1                                              \\
9                       & 0,8                                            \\ \bottomrule
\end{tabular}
\caption{Pastatykkelser}
\label{pastatykkelser}
\end{table}

Kilde: Marcato Atlas 150 brukermanual
