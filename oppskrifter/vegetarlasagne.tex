\section{Vegetarlasagne}
\label{Vegetarlasagne}

Antall porsjoner på formen `fire porsjoner'\\
Hvor lang tid det tar\\
Pris

\paragraph{Ingredienser---Grønnsakssaus}
\begin{itemize}[noitemsep]
	\item Olivenolje
	\item 1 rød chili
	\item 3 gulrøtter
	\item 1 stor rød paprika
	\item 1 $\sfrac{1}{2}$ squash, i halvmåner
	\item 1 boks (400 g) hermetiske tomater, hakket
	\item 1 dl kremfløte
	\item $\sfrac{1}{2}$ ts salt
	\item $\sfrac{1}{2}$ ts sukker
	\item Sort pepper
	\item Frisk basilikum, grovt hakket
\end{itemize}

\paragraph{Ingredienser---Mellom lagene}
\begin{itemize}[noitemsep]
	\item 300 gram frisk spinat
	\item 200 gram feta
	\item ca. 12 lasagneplater
	\item 150 gram revet Norvegia
\end{itemize}

\paragraph{Framgangsmåte}
\begin{enumerate}[noitemsep]
	\item Skjær opp alle grønnsakene
	\item Finn frem en dyp stekepanne eller gryte og fres chilien i litt olje. Tilsett
	deretter gulrøtter, paprika og squash, som får surre med i et par minutter.
	\item Ha i hermetiske tomater og kremfløte, og smak sausen til med sukker, salt og pepper.
	\item La sausen koke på lav varme i 10--15 minutter, og tilsett finsnittet basilikum	helt til slutt.
	\item Rens og skyll spinaten og fres den deretter i litt nøytral olje til den så vidt
	faller sammen. Krydre med salt og pepper. Hell av eventuell væske.
	\item Finn frem en ildfast form, ca. 30*20 cm. Legg grønnsakssaus, fyll og
	lasagneplater lagvis, for eksempel i denne rekkefølgen: Fordel 1/3 av
	grønnsakssausen i bunnen av formen og fordel deretter halvparten av spinaten over.
	\item Smuldre halvparten av fetaosten over og dekk deretter med ett lag lasagneplater.
	\item Gjenta samme prosedyre en gang til, avslutt med resten av grønnsakssausen og ett
	lag med lasagneplater oppå. Dytt lasagneplatene litt nedi sausen og topp det
	hele med revet ost.
	\item Stek lasagnen ved 200 grader (over- og undervarme) på nederste rille i ca. 30
	minutter, til den er gyllen og fin oppå. Serveres med en enkel grønn salat og
	for eksempel baguetter som tilbehør.
\end{enumerate}

Tips: Den friske spinaten kan eventuelt erstattes med frossen spinat.\\
    Lasagnen bør nytes samme dag da spinaten ikke bør varmebehandles mer enn en gang. Les mer om det her. \\

Kilde:\href{http://trinesmatblogg.no/2015/03/05/vegetarlasagne-med-spinat-og-feta/}{Trines Matblogg - Vegetarlasaagne}
