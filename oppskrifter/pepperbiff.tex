\section{Pepperbiff med peppersaus}
To porsjoner\\
Tid 40 min

\paragraph{Ingredienser}
\begin{itemize}[noitemsep]
	\item 350 gram  pepperbiff
	\item 1 pakke Toro Peppersaus eller lag egen\ref{peppersaus}
	\item 1 terning Kjøttbuljong
	\item 2 dl Kremfløte
	\item Salt og pepper
\end{itemize}

\paragraph{Tilbehør}
\begin{itemize}[noitemsep]
	\item Sjampinjonger (14stk små per pers)
	\item 2 dl Ris, se oppskrift\ref{ris}
	\item Smørdampet brekkoli, se oppskrift\ref{brokkoli}
	\item Fløtegratinerte poteter, se oppskrift\ref{flotegratinerte}
\end{itemize}

\paragraph{Framgangsmåte}
\begin{enumerate}[noitemsep]
	\item Ta biffen ut og la den ligge i romtemperatur i 30 minutter
	\item Vask risen og la den stå og trekke mens du venter på biffen
	\item Tøm posen med saus i en panne og tøm oppi fløte, vann og buljong
	\item Kutt opp sjampinjong og ligg den på en stekepanne med smør
	\item Slå på risen og sjampinjongpannen og sausen
	\item Ta smør i stekepannen til biffen og la det bli brunt før du hiver biffene på
	\item Snu biffen ofte, krydre med salt og pepper. Stek til det er motstand i biffen og det pipler ut saft
	\item La risen koke til du ser vannet er stort sett vekke, sjekk at den er klar
\end{enumerate}

Kilde: Pappa/Frode Lindseth

Tips: Server Panna Cotta til desert\ref{pannacotta}
