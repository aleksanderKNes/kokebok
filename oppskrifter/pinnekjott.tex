\section{Pinnekjøtt}
Åtte porsjoner

\paragraph{Ingredienser}
\begin{itemize}[noitemsep]
  \item 3 kg kokefaste poteter blir ca 3 poteter per person
  \item 3 poteter ekstra til kålrabistappen
  \item 1 gulerot
  \item 4.5 kg kålrabi / ca 4 store
  \item 400 gram pinnekjøtt per pers (Handles hos meny eller Brakstads Etterfølgere (slakter på bystasjonen) eller hos slakteri i godvik)
  \item Bjørkepinner
  \item Evt desert
  \item 3dl kremfløte
  \item Tre store panner
  \item Serveringsskåler
\end{itemize}

\paragraph{Framgangsmåte --- dagen før}
\begin{enumerate}[noitemsep]
  \item Skrell potetene (ingen øyner),for at de ikke skal bli brune må de ligges i en panen og tildekket med vann
  \item Skrell kålrabien, og kutt i terninger, ligg disse også i en panne og tildekk med vann
  \item Lag klar deserten
  \item Legg noen lager med bjørkepinne i bunnen av pannen til pinnekjøttet og legg pinnekjøttet oppi. Dekk alt med vann og la det stå over natten for å unngå at det blir alt for salt
\end{enumerate}

\paragraph{Framgangsmåte --- dagen}
\begin{enumerate}[noitemsep]
  \item Om morgenen, bytt vann på pinnekjøttet. Hell av det gamle vannet og tilsett nytt.
  \item Når det nærmer seg tre timer før maten skal serveres, hell av alt vannet og hell på vann så bjørkepinnene dekkes i bunnen, men at ikke pinnekjøttet kokes.
  \item Tre timer før maten skal serveres må pannen med pinnekjøttet slåes på så de får dampe i tre timer, til kjøttet har løsnet fra beinet.
  \item Kok opp med lokket på, og skru ned så det bare akurat koker. Sørg for at man ikke tørrkoker, da blir det dårlig smak
  \item Sjekk med jevne mellomrom (VIKTIG)
  \item Øs av pinnefettet fra toppen når pinnekjøttet er ferdig og ha det i en liten panne på varme, til maten er klar for å serveres.

  \item En time før servering, Ha litt salt i vannet?  slå på platen med kålrabistappen. La den koke til kålrabien er helt mør (myk, som kokte poteter), omtrent 30 minutt.
  \item Når de er ferdigkokt: ha oppi en kokt gulrot og 2--3 poteter for en mildere smak
  \item Smak til med kremfløte og en liten ause pinnefett

  \item Potetene skal koke som vanlige poteter, i en 20 minutter, til de er mør.

  \item Server på varme tallerkner
\end{enumerate}

\subsection{tips}
Hvis man kjøper pinnekjøtt som er saget i to så får man plass til mer i pannen
Server gjerne panna cotta til desert, se oppskrift-\ref{pannacotta} for oppskrift

\subsubsection{Notater}
Julen 2016: En tredjedel av potetene var igjen, en fjerdedel av stappen og ingen kjøtt. Skulle smeltet smør til Vibecke
Tre personer brukte en time på å skrelle poteter og kålrabi for 10 personer.


\paragraph{Julebord hos Heine}
Åtte personer.
400gram pinnekjøtt * 8 personer = 3.2kg * 360kr/kg = 1150,-
500kr i kremfløte, vaniljestang og bringebær
50kr kålrabi
50kr i poteter
Totalt circa 1750,-/ 8 = 220,- per pers


Endte opp med:
500g Røykt Vossakjøt pinnekjøtt fra Meny /per person til 900.3 kr (359kr/kg)
4.5kg kålrot 44kr
3.1kg poteter 45kr
+ Pannacotta til alle
Totalt 150kr per pers


\paragraph{Et annet år}

Kålrabi 6,496 25,35,-
Vaniljestang 5 105,-
Fløte 5 stk 0,3L 89,5,-
gulrøtter 1pk 25,-
Pinnekjøtt 958,5,-
Bringebær 2 poser 29,-
Poteter 4kg 40,-
gelatin 10
Potetskreller 2 89,8,-
melis 8,-


Total
1380,15
